% !TEX root = paper.tex

\section{Introduction}
\label{sec:intro}
Blood glucose concentration plays an important role in personal health.
Hyperglycemia (high blood glucose level) results in diabetes, leading to health risks such as pancreatic function failure, immunity reduce and ocular fundus diseases~\cite{bib:DC95:Klein}.
Meanwhile, hypoglycemia (low blood glucose level) also brings complications such as confusion, shakiness, anxiety, and even coma or death if not treated in time~\cite{bib:AJM91:DCCT}.
%The International Diabetes Federation (IDF) reports that there were 415 million diabetic patients in 2015 and the number will rise to 642 million by 2040~\cite{bib:IDF15}.
People with diabetes need tight control of their blood glucose concentration to avoid both short-term and long-term physiological complications.
While non-diabetic people normally have adequate self-regulation of blood glucose concentration, they still have the risk of hypoglycemia when taking prolonged exercises, drinking excess amounts of alcohol, having eating disorders, taking certain medicines (\eg certain painkiller and antibiotic), or having pre-diabetes~\cite{bib:JCEM13:Eckert} \cite{bib:NEJM82:Felig}.

Although continuous blood glucose monitoring is crucial for blood glucose management and beneficial for hyper- and hypoglycemia warning, it can be invasive and inconvenient, especially during daily life.
A standard blood glucose measurement is to collect and analyze a drop of blood by finger pricking, which requires a new prick on the finger for every observation.
Alternatively, non-invasive (without penetrating the skin) continuous glucose monitoring (CGM) has attracted extensive research leveraging techniques such as thermal infrared spectroscopy, Raman spectroscopy and impedance spectroscopy~\cite{bib:JDST10:Vaddiraju}.
However, most CGM devices are expensive, cumbersome to wear for extended time, complicated in terms of operation/maintenance, making them unattractive for both diabetic patients and non-diabetic people.

Towards more ubiquitous blood glucose monitoring when traditional CGM devices are unavailable or inconvenient to wear, researchers propose to explore the increasingly rich sensors embedded in commercial fitness wearables and smartphones as a complement.
In addition to the inherent glucose metabolism, blood glucose also correlates to easily measurable physiological activities such as food and drug intake, energy expenditure, sleep quality and emotional states~\cite{bib:DRCP15:Iwasaki}.
Pioneer works~\cite{bib:EMBC12:Nguyen} \cite{bib:SEMPER16:Ranvier} \cite{bib:JDST14:Sobel} have proposed preliminary systems adopting bio-sensors (\eg ECG electrodes) and fitness bands (\eg accelerometer and galvanic skin response sensors) to predict blood glucose concentrations and alarm abnormal blood glucose events.
Nevertheless, they are validated with limited number of measurements~\cite{bib:SEMPER16:Ranvier} \rev{and for short-term (\eg 3 hours without the need of CGM devices~\cite{bib:JDST14:Sobel})} or still require complex multi-sensory platforms~\cite{bib:EMBC12:Nguyen} \cite{bib:JDST14:Sobel}.

In this work, we design \sysname, a personalized smartphone-based non-invasive blood glucose monitoring system that detects abnormal blood glucose events by jointly tracking meal, drug and insulin intake, as well as physical activity and sleep quality.
\rev{\sysname is designed as a \textit{temporal} alternative for CGM devices when they are uncomfortable or inconvenient to wear.
It monitors blood glucose levels every 3 minutes and can be used for two to three weeks before re-calibration using the CGM devices.
The blood glucose levels tracked by \sysname help both diabetic patients and non-diabetic people who want to track their blood glucose to adjust their lifestyle as needed (\eg exercise more and take less carbohydrate) and take other precaution measures when an abnormal blood glucose level is detected.}
\sysname comprehensively considers \textit{generic}, \textit{grouped} and \textit{user-specific} correlations between blood glucose levels and the measurable external factors, which are largely overlooked in previous works.


The key challenge in designing \sysname is to learn effective, accurate and personalized blood glucose models.
While there have been general blood glucose models characterizing universal trends between blood glucose concentration and various external factors~\cite{bib:IJNMBE16:Oviedo}, they have to be adjusted based on user-specific data to account for inter-user differences~\cite{bib:ICMLA13:Bunescu}.
Yet it is often difficult to collect sufficient data to directly build up personalized models~\cite{bib:KDHealth16:Marling}:
\emph{(i)}
A disposable enzyme of glucose sensor embedded in the CGM device is only capable of a few days \cite{bib:CGM_wiki} \cite{bib:CGM_wave}, and most users are unwilling to wear CGM devices frequently due to discomfort.
\emph{(ii)}
Despite their importance, measurements of hyper- and hypoglycemia events are rare compared with normal blood glucose concentrations, making it difficult to accurately detect abnormal blood glucose events.

To take full advantages of the \textit{sparse}, \textit{imbalanced} measurements to build \textit{personalized} blood glucose level models, we first conduct feature extraction from both physiological and temporal perspectives.
We propose \modelname (multi-division deep dynamic recurrent neural network), an efficient learning paradigm that extracts general blood glucose level relevant features and preserves user-specific characteristics.
\modelname advances previous personalized recurrent neural network (RNN) structures via a group-shared input layer to extract distinctive feature representations within the same group (\ie non-diabetic, type I and type II diabetic).
In short, \modelname can be regarded as both a deep hierarchical RNN architectures, \rev{which is a combination of single user division and grouped user division learning}.
Evaluations on the blood glucose dataset composed of 112 users \rev{collected for $6$ to $30$ days} show that \modelname outperforms both generic learning (\ie ignoring inter-user differences) and personalized learning (due to lack of measurements), and also achieves notably higher inference accuracy than conventional shallow learning algorithms.

The key contributions of this work are summarized as follows.
\begin{itemize}
  \item
  To the best of our knowledge, \sysname is the first smartphone-based personalized blood glucose monitor that works without CGM devices \rev{for an extended duration of time.}
  It automatically collects daily exercise and sleep quality, and infers the current blood glucose level of users, together with manual records of food, drug and insulin intake \rev{and only needs re-calibration using CGM devices after two or three weeks.}
  \item
  We propose \modelname, an efficient multi-division deep dynamic RNN framework for blood glucose level inference.
  \rev{The newly designed grouped input layers, together with the adoption of a deep RNN model,} offer an opportunity to build blood glucose models for the general public based on limited personal measurements from single-user and grouped-users views.
  \item
  We conduct extensive evaluations on both diabetic patients and non-diabetic people.
  \rev{Experimental results from a dataset of 35 non-diabetic people, 38 type I and 39 type II diabetic patients collected for at least 6 days per person demonstrate that \sysname yields an average accuracy of 82.14\%, and outperforms traditional general learning, group-level learning, personalized learning and shallow/deep learning algorithms in precision and recall.}
\end{itemize}

In the rest of this paper, we review related works in \secref{sec:relwork} and present an overview of \sysname in \secref{sec:overview}.
The design and evaluation of \sysname are detailed in \secref{sec:design} and \secref{sec:eval}, respectively.
Finally, we conclude in \secref{sec:conclusion}.

