
\section{Design}
\label{sec:design}
This section presents the detailed design of \sysname.

\subsection{External Factor Sensing}
\label{subsec:external}
It is infeasible to record a complete list of impacting factors on blood glucose concentration.
In \sysname, we measure five major external factors including food, drug and insulin intake, as well as physical activities and sleep quality.

\fakeparagraph{Food Intake}
As food intake is a main source of carbohydrate, \sysname provides a food menu for users to input their meals.
We categorize meals into five types based on the carbohydrate food list~\cite{bib:carblist}, including grains, vegetables, milk and egg, fruits and meats.
Users are asked to enter the food types and amounts of their meals and \sysname calculates the carbohydrate of a meal based on this information.

\fakeparagraph{Drug Intake}
Oral diabetic drugs enhance the secretion of insulin into the blood and are usually used for type II diabetic patients.
In \sysname, a drug menu of common 11 oral diabetes is presented for users to input their drug intake based on \cite{bib:druglist}.
Users are required to select the drug name and record the drug dosage into \sysname.
Then \sysname transforms the drug intake into quantitative effects on blood glucose based on physiological models~\cite{bib:AIM07:Bolen}.

\fakeparagraph{Insulin Injection}
Inulin injection is widely used for blood glucose control for type I and type II patients. 
\sysname provides an insulin type list based on~\cite{bib:insulinlist} for users to record the usage and dosage of their insulin injection.
We transform insulin injection into insulin dynamics in the physiological model (detailed later).
 

\fakeparagraph{Physical Activity}
Daily activities \eg exercises consume the carbohydrate and affect blood glucose levels.
In \sysname, we adopt an efficient activity recognition scheme~\cite{bib:KDDEN11:Kwapisz} to automatically record six common physical activities (walking, running, going upstairs, going downstairs, sitting and standing) along with the corresponding durations.
\sysname then calculates the caloric expenditure based on the widely used calorie calculator~\cite{bib:HealthStatus, bib:CalorieCounter}.
\begin{equation}\label{eq:calorie_burn}
  Calorie Burn = (BMR/24)*MET*T,
\end{equation}
where BMR (Basal Methobolic Rate) is the amount of energy required to simply sit or lie quietly, and MET (Metabolic Equivalent) is the ratio of the work metabolic rate to the resting metabolic rate.
\sysname finally leverages the calories as input for our blood glucose level model.

\fakeparagraph{Sleep Quality}
Sleep quality has a long-term influence on the blood glucose level~\cite{bib:DRCP15:Iwasaki}.
\sysname automatically measures sleep quality using smartphones as in~\cite{bib:UbiComp14:Gu}.
The output sleep quality score is then used in our blood glucose level model.

\subsection{Feature Engineering}
\label{subsec:features}
We extract features from both the physiological view and the temporal view to infer blood glucose levels.

\subsubsection{Features from Physiological View: $F_p$}
\label{subsubsec:physiological}

\begin{figure}[h]
  \centering
  \includegraphics[width=0.9\columnwidth]{./img/Physiological_correlation1.pdf}
  \caption{Temporal graph of the physiological model.}
  \label{fig:phymodel}
\end{figure}

In \sysname, we use smartphone to collect the external factors $U_t=\{U_c(t),U_e(t),U_s(t), U_i(t)\}$, and apply the physiological model to generate real-time observed vector $X_{t}=\{C_{g1}(t), C_{g2}(t),I_{s}(t),I_{m}(t), I_{a}(t), I(t),\\ E(t), S(t),  G_{m}(t), G(t)\}$ for every blood glucose sample at the corresponding time $t$.
The hidden physiological factors at $t+1$ can be calculated by $X_{t+1}=f(X_t, U_t)$, where $f$ is the station transition functions.
\sysname computes $X$ at each time step and treats it as 10-dimensional features.

The feature engineering of physiological view is to leverage a physiological model to quantify the dynamics of physiological factors in the body. 
The physiological model, based on the physiology mechanism of blood glucose in Section~\ref{sec:preliminary}, has been widely studied in previous works \cite{bib:briegel2002nonlinear,bib:duke2010intelligent, bib:plis2014machine}.
It primarily measures the real-time values of carbohydrate, insulin and glucose influenced by the external factors. 
We constructed our physiological model based on the work \cite{bib:duke2010intelligent}, with an extension of sleeping fact according to its physiological impact discussed in Section~\ref{sec:preliminary}.

\paragraph{Temporal Graph of Physiological Model}
The physiological model of \sysname describes the physiological factors from five aspects: carbohydrate dynamics, insulin dynamics, exercise dynamics, sleep dynamics and blood glucose dynamics.
%Given the hidden physiological vectors $X_{t}=\{C_{g1}(t), C_{g2}(t),\\ I_{s}(t),I_{m}(t), I_{t}, G_{m}(t), G(t), E(t)\}$, where the elements of this vector represent
%the hidden physiological factors at time point $t$, and the observable input vector $U_{t}=\{U_{c}(t), U_{e}(t), U_{s}(t), U_{i}(t)\}$ , in which $U_{c}$ stands for the
%carbohydrate proportion of meals, $U_{e}$ indicates the calories cost by exercise,  $U_{s}$ is the sleep score and $U_{i}$ states for the amount of insulin injected or simulated
%by the diabetes drugs  at time point $t$. In particular, the sleep quality $U(s)$ is a constant during a whole day.
%The station transition functions can be represented as $X_{t+1}=f(X_t, U_t)$, and its corresponding
%temporal transition graph is shown as \figref{fig:phymodel}.  %\tabref{phy_tab} details the transformation functions of the carbohydrate $C_{g1}$ and $C_{g2}$, and insulin
%As \figref{fig:phymodel} shown, the equations of $C_{g1}$ (Equation~\ref{Eq:Cg1}) and $C_{g2}$ (Equation~\ref{Eq:Cg2}) indicating the temporal transitions of carbohydrate consumption and the carbohydrate digestion respectively.

%Positive factors of physiological model indicate the carbohydrate absorption and the hepatic glucose production, which will increase the blood glucose level.
%Negative factors of  physiological model describe the insulin independent uptake, insulin dependent uptake and the renal clearance. We will detail these factors as follows:


\emph{Carbohydrate dynamics}: 
Carbohydrate dynamics refer to the transitions of carbohydrate consumption $C_{g1}$ and the carbohydrate digestion $C_{g2}$.
Equation~\ref{Eq:Cg1} and Equation~\ref{Eq:Cg2} show their transition equations respectively, where $U_{c}$ stands for the carbohydrate proportion of meals.
\begin{equation}\label{Eq:Cg1}
C_{g1}(t+1)=C_{g1}(t)-\alpha_{1}^c*C_{g1}(t)+U_{c}(t)
\end{equation}

\begin{equation}\label{Eq:Cg2}
C_{g2}(t+1)=C_{g2}(t)+\alpha_{1}^c*C_{g1}(t)-\alpha_{2}^c*C_{g2}(t)
\end{equation}

\emph{Insulin dynamics}: Insulin dynamics indicates the transitions of subcutaneous insulin absorption $I_{a}$ (Equation~\ref{Eq:Is}),
the insulin secretion by pancreas $I_{s}$ and  the insulin mass $I_{m}$ (Equation~\ref{Eq:Im}). The level of active plasma insulin $I$ (Equation~\ref{Eq:I}). $U_{I}$
states for the amount of insulin injected or simulated by the diabetes drugs. $S^I$ and $bm$ refer to the insulin sensitive
and body mass respectively.

\begin{equation}\label{Eq:Is}
I_{a}(t+1)=I_{a}(t)-\alpha_{f,r,m}^I*I_{a}(t)+U_{i}(t)
\end{equation}


\begin{equation}\label{Eq:Isec}
    I_{s}(t+1)=
   \begin{cases}
    I_{s}(t)+min[\alpha_1^I(\alpha_2^I(G_t-G_0))+\alpha_3^I*G_0, \triangle I_{max}^{s}] &\mbox{Type II}\\
    I_{s}(t)+0 &\mbox{Type I}
   \end{cases}
  \end{equation}

\begin{equation}\label{Eq:Im}
I_{m}(t+1)=I_{m}(t)+\alpha_{f,r,m}^I*I_{a}(t)+\alpha_{a}^I*I_{a}(t)-\alpha_c^I*I_{m}(t)
\end{equation}


\begin{equation}\label{Eq:I}
I(t)=\frac{I_{m}(t)*S^I}{142*bm}
\end{equation}

\emph{Exercise dynamics}: Exercise dynamics $E$ denotes the exercise effect on insulin over the past time window. This
long-term influence can be expressed by a cumulative moving average \cite{bib:lowry1992multivariate, bib:cma} as Equation~\ref{Eq:E} in the physiological model.

\begin{equation}\label{Eq:E}
E(t-t_0+1)=(t-t_0)*E(t-t_0)+U_{e}(t-t_0)
\end{equation}

where $t$ and $t_0$ are the current and beginning time point in the past time window. In \sysname,
the window size of exercise is set to 24 hours, which optimizes the experimental results and matches the conclusion
of clinical studies.

\emph{Sleep dynamics}: Sleep dynamics $S$ represents the sleeping quality effect on insulin. In physiological model, sleeping
effect also has a long-term influence on the insulin as the exercise effects. Specifically, it maintains a constant effect on
blood glucose for each day. Equation~\ref{Eq:S} shows its transition equation.

\begin{equation}\label{Eq:S}
S(t-t_0+1)=(t-t_0)*S(t-t_0)+U_{s}(t-t_0)
\end{equation},

where $t$ and $t_0$ are the current and beginning time point in the past time window.
In \sysname, the window size of sleep lasts for 7 days, which optimizes the experimental results and matches the conclusion
of clinical studies.


\emph{Blood glucose dynamics}: Blood glucose dynamic points to the fluctuation of glucose mass $G_m$  in Equation~\ref{Eq:Gm},
and glucose concentration $G(t)$ in Equation~\ref{Eq:G}.
\begin{equation}\label{Eq:Gm}
G_m(t+1)=G_m(t)+\delta_{abs}+\delta_{egp}-\delta_{ind}-\delta_{dep}-\delta_{clr}
\end{equation}
\begin{equation}\label{Eq:G}
G(t)=G_m(t)/(2.2*bm)
\end{equation}

Specifically, $\delta_{abs}$ in Equation~\ref{Eq:abs} refers to the impact results of carbohydrate absorption, and
$\delta_{egp}$ in Equation~\ref{Eq:egp} indicates to the hepatic glucose production from
the liver. These two factors improve the blood glucose concentration of body.
\begin{equation}\label{Eq:abs}
  \delta_{abs}=\alpha_3^c*\alpha_2^c*C_{g2}
\end{equation}

\begin{equation}\label{Eq:egp}
  \delta_{egp}=\alpha_2^{egp}*exp(-I(t)/\alpha_3^{egp})-\alpha_1^{egp}*G(t)
\end{equation}

$\delta_{ind}$ (Equation~\ref{Eq:ind}) describes results of insulin independent uptake, which is consumed by the central nervous
system and the red blood cells.
$\delta_{dep}$ (Equation~\ref{Eq:dep}) indicates the impact results of insulin dependent uptake. It reflects the effects
of insulin promoting muscle cells and fat cells to absorb glucose, which combines the influence of sleeping and exercise factors.
$\delta_{clr}$ (Equation~\ref{Eq:clr}) stands for the influence of renal clearance on blood glucose. Once blood glucose
concentrations exceeds the renal clearance threshold $\tau$, the kidneys begin to remove excess glucose from the blood.
The three factors decrease the blood glucose level.

\begin{equation}\label{Eq:ind}
  \delta_{ind}=\alpha_1^{ind}/\sqrt{G(t)}
\end{equation}

\begin{equation}\label{Eq:dep}
  \delta_{dep}=\alpha_1^{dep}* E(t)*S(t)*I(t)/(G(t)+\alpha_2^{dep})
\end{equation}

\begin{equation}\label{Eq:clr}
  \delta_{clr}=\alpha_1^{clr}*(G(t)-\tau)
\end{equation}

The default values of parameters in the physiological model are set
based on \cite{bib:duke2010intelligent}. They are further tuned for each person
by the classifier on the training dataset by 10-cross validations, which optimize
the experimental results.

In \sysname, we use smartphone to collect the external factors $U_t=\{U_c(t),U_e(t),U_s(t), U_i(t)\}$, and
apply the physiological model to generate real-time observed vector
$X_{t}=\{C_{g1}(t), C_{g2}(t),I_{s}(t),I_{m}(t), I_{a}(t), I(t),\\
 E(t), S(t),  G_{m}(t), G(t)\}$ for every blood glucose sample at the corresponding time $t$.
The hidden physiological factors at $t+1$ can be calculated by
$X_{t+1}=f(X_t, U_t)$, where $f$ is the station transition functions.
\sysname computes $X$ at each time step and treats it as 10-dimensional
features.

\subsection{Features from Temporal View: $F_t$}
The blood glucose dynamic holds a natural temporal ordering, \sysname mines its timing characteristics
to predict the blood glucose level.

Three dimensional features of temporal view have been considered.

\paragraph{Historical Blood glucose trend: $F_{t1}$}
Since the trend of blood glucose of a single person hardly occurs significant alteration
within a short period, \sysname calculates the historical blood glucose trend $G_{Trend}$ by average the true blood glucose
concentrations at each corresponding time stamp $t$ over recent $D$ days in the past by Equation~\ref{Eq:his_trend}.

\begin{equation}\label{Eq:his_trend}
  G_{Trend}=\frac{1}{D}\sum_{d=1}^{D}G(f_t(d)),  t=1,2,...,N
\end{equation}

where $N=480$ refers to the number of time stamps in a day, and $D$ reflects the days of measurement. $G(f_t(d))$ indicates the true blood glucose value measured by the CGM
at the time stamp $t$ of the $d$th day. In \sysname, we select $D=5$ based on the optimal experimental results

\sysname treats it as a temporal feature to reflect the historical blood glucose trend
over the same period.

\paragraph{Blood glucose value with similar physiological features: $F_{t2}$}
As the blood glucose concentrate is determined by the factors of carbohydrate, insulin, exercise intensity and sleep quality, the similar blood
glucose values should maintain similar values of these physiological factors. Accordingly, \sysname applies the k-Nearest Neighbors algorithm \cite{bib:KNN}
to search for 5 blood glucose values with the most similar physiological features, and average them as one dimensional feature.

\paragraph{Recent physiological features: $F_{t3}$}
Considering the physiological features may generate the temporal delay effects on the blood glucose, \sysname also takes the physiological factor vectors
in the past 15 minutes as recent physiological features to measure their impacts on the current blood glucose concentration.

\sysname takes the physiological-temporal features as input, and predict the blood glucose level by \modelname.


\subsection{Blood glucose level prediction}

% !TEX root = paper.tex

\subsection{Intuition}
Given the features extracted from the physiological process, it seems plausible to perform any classification algorithm for blood glucose level prediction. Nonetheless, this plug-and-play approach will neglect important information from (1) dynamics of the process, and (2) inter-user similarity among same group of participants. Traditionally, various sequential classification methods[], e.g., hidden Markov model (HMM), recurrent neural networks (RNN), and dynamic conditional random fields (CRF), are used to capture the temporal correlation of the input feature. The inter process correlations are often times incorporated with the so-called multi-task learning approaches[], which learns processes (or tasks) in parallel to improve classification or to reduce the data sample requirement. 

In this paper, a novel machine learning paradigm, namely Multi-division deep-dynamic RNN (Md$^3$RNN), is proposed. To include the the aforementioned information sources in an unified framework, we develop two key ideas that extend the classical RNN. Firstly, the single hidden layer in RNN is replaced with several deep stacked layers. The deep structure in the new model is able to describe complex, mutli-scale system dynamics that would otherwise be ignored (or averaged out) by prior ``shallow'' models such as HMM, RNN, and CRF. Secondly, the correlations among users, being quite significant within user groups (divisions), are encoded by group-shared input layer and common hidden layers, whereas the distinct characteristics of individual users are modeled with  different output layers for personalized prediction. Within a larger scope of machine learning, the proposed Md$^3$RNN aims to leverage recent advancement of deep learning and multi-task learning, to model group-interacted time series data having complex temporal dynamics. It can be viewed as both a deep extension of RNN, and an intermediate between single-task learning and multi-task learning, hence the name Md$^3$RNN.

The overall configuration of the proposed model is summarized in Fig.\ref{fig:rnn}. Detailed construction of each component is given in the sequel.     

\begin{figure}[!t]
  \centering
  \includegraphics[width=0.9\columnwidth]{./img/pics_RNN.pdf}
  \caption{The Md$^3$RNN structure}
  \label{fig:rnn}
\end{figure}

\subsection{Model construction by layers}
The input of the Md$^3$RNN are the features extracted from the physiological model. The labeled data sequences for user number $j$ at time $t$ are denoted by $(x_i^{j},y_i^{j})$. We also adopt an index set convention, that $(x_A^{B},y_A^B)$ represents the data set $\left\{(x_i^{j},y_i^{j}) | i \in A, j\in B\right\}$ given index sets $A$ and $B$.
\subsubsection{Grouped Input Layer}
In the context of blood glucose prediction, available inputs are naturally divided into three groups according to the health condition of the participant from whom the data was generated. Notation-wise, we utilize $H$, $I$ and $II$ to indicate the the group of healthy user, user with type I diabetes and those having type II diabetes, respectively. Since the extracted features are essentially physiological indexes of an ``average'' person, they must go though different transformations to represent useful information of three distinct groups. This consideration motivate the design of the input layer (bottom of Fig.\ref{fig:rnn}) - it is divided into three units that performs different linear and non-linear transformation according to user groups. For instant, a data sample $x_t^{I_j}$, generated at time $t$ from the $j^{th}$ user of type I, undergoes the following processing:
\begin{align}
\tilde{x}_t^{I_j} = \sigma \left( W^Ix_t^{I_j} \right)
\end{align}     
where $W^I$ is the coefficients of the affine transformation \footnote{We assume that the interception is included in $W$. This can be done by simply adding a feature of all $1$s.}, $\sigma$ is some activation function, and $\tilde{x}_t^{I_j}$ is the output of the input layer for that data sample. Similar operations are conducted for data samples from group $H$ and $II$, but with different transformation coefficients. Intuitively, the shared transformation within groups would improve the learning of parameters (vs. single task learning), as information from all data in a homogeneous group is used.  

\subsubsection{Deep Dynamic Layer}
A common hidden layer is designated to capture the dynamics of the blood glucose evolution process. The underlying assumption is that, the biological and chemical reactions governing blood glucose variation are similar for all people, despite of grouped behaviors in the representation of physiological indexes (input layer), or individual characteristics in exhibited glucose level. This assumption could be justified by a series of medical research[][]. Moreover, since all users share the same hidden layer, all collected data samples are eventually helping the estimation of its parameters. The availability of rich information for the hidden layer makes the learning of a deep structure possible. In Md$^3$RNN, a number of Long Short Term Memory (LSTM) networks are stacked together (middle of Fig.\ref{fig:rnn}), to increase the overall model flexibility. In particular, it has been justified in both theory and practice that stacked LSTMs are able to capture dynamics occurring at different time scales, which in the current application would enable the modeling of both slow and rapid biological/chemical reactions. Mathematically, given the output from the grouped input layer, the deep dynamic layer performs
\begin{align}
h^0_t=
\end{align}



\subsubsection{Personalized Output Layer}


\subsection{Cost Sensitive Learning and Hyperparamter selection}
