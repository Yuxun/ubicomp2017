
\section{Preliminary}
\label{sec:preliminary}
 Blood glucose indicates the amount of sugar present in a personal body. It usually comes from the food intake and is the primary source of energy utilized by body.
 \subsection{Physiology of Blood Glucose }
 Glucose is transported from the intestines or liver to body cells via the bloodstream, and is made available for cell digestion via the hormone insulin, released by the pancreas. While the glucose level rising, the pancreas produces insulin to promote the transfer of glucose from the blood to the cells of muscle and fat, and to increase the uptake of glucose into the liver. While the glucose level decreasing, the insulin level drops as well and the liver increases glucose production  to bring the blood glucose level back to normal.

 Normally, a health individual body can properly regulate the blood glucose level with a range of 4.4 mmol/L to 6.1 mmol/L \cite{bib:blood_sugar}, and slightly grows up 7.8 mmol/L after eating. For those whose blood glucose level keeping above 7.8 mmol/L for a prolonged period, they take the risks of diabetes mellitus, and should take hypoglycemic drugs or inject insulin to control the blood glucose. If the blood glucose level descends under 4.4 mmol/L, it is a sign of hypoglycemia. Accordingly, we classify the blood glucose values into four levels in Table~\ref{blood_glucose_levels}.

\begin{table}[]
\centering
\caption{Blood Glucose Levels}
\label{blood_glucose_levels}
\begin{tabular}{|c|c|c|}
\hline
\textbf{Blood Glucose Value (mmol/L)} & \textbf{Glocose Level} & \textbf{Explanation}                      \\ \hline
(0, 4.4{]}                            & Level 1                & Low Blood Glucose                          \\ \hline
(4.4, 6.1{]}                          & Level 2                & Normal Level of Fasting Blood Glucose      \\ \hline
(6.1, 7.8{]}                          & Level 3                & Normal Level of Postprandial Blood Glucose \\ \hline
(7.8, $+\infty$)                               & Level 4                & High Blood Glucose                         \\ \hline
\end{tabular}
\end{table}

\subsection{Impact Factors of Blood Glucose}
 Blood glucose level is mainly determined by two dimensional factors:  the internal factors and the external factors. The internal factors indicates the genetics of persons. Many clinic researches or reports have demonstrated genetics are associated with diabetes \cite{bib:simpson1978genetics,bib:diabetes_co_uk,bib:rotter1984genetics,bib:concannon2009genetics,bib:salopuro2004common}.
 It is different from individuals. The external factors are mainly composed of food intakes, the exercise, the sleep quality and drugs or insulin inputs~\cite{bib:duke2010intelligent}. The physiological mechanism of external factors on blood glucose are shown as follows.

\paragraph{Food intake} The carbohydrates of the food intake can be quickly absorbed by gut and turned into the blood glucose. A steep rising trend usually occurs after consuming high-carbohydrate meals. Naturally, high-carbohydrate foods (\eg white grain products, bread and cookies) contributes higher blood glucose levels than low/moderate-carbohydrate foods (\eg meat, vegetables).
Hence, the food intake is an important external factor of blood glucose.
\paragraph{Exercise} Exercise impacts the blood glucose level by the energy consumption. The movement of muscles will trigger the body cells absorbing sugar from blood for energy of doing exercise.  By doing exercises, physical activity can help lower the blood glucose level for several hours after stop moving. The regular exercises promote the sensitivity of cells to the insulin, which can help keep blood glucose level vary in a normal range.

\paragraph{Sleep Quality} Sleep quality is strongly linked to the blood glucose levels. Several clinical studies \cite{bib:scheen1998roles,bib:scheen1996relationships,bib:spiegel2005sleep} have demonstrates poor sleep quality is likely to results in high glucose levels. The body's reaction to sleep loss can resemble insulin resistance, a precursor to diabetes. Under this case, the cells of body easily fail to generate hormone efficiently, resulting in high blood glucose therefore. Moreover, sleep disorder is a key culprit of overweight \cite{bib:punjabi2002sleep, bib:knutson2006role}, which is also closed associated with the diabetes.

\paragraph{hypoglycemic drugs} The hypoglycemic drugs are mainly used for the patients with type II diabetes\cite{bib:eurich2007benefits,bib:jung2006antidiabetic,bib:patel2012overview}. Usually, the anti-diabetes medication control the blood glucose concentrations by three ways: (1) stimulating the insulin secretion from pancreas, (2) promoting the tissue sensitivity to insulin, (3) low the rate at which glucose is absorbed from the gastrointestinal tract.

\paragraph{Insulin} Insulin injection is a common way to treat the diabetes. For those who suffer Type I diabetes, they have to inject insulin every day due to their bodies fail to release insulin. For those who have Type II diabetes, insulin injection is also required when your body cannot proper react to insulin or produce enough of it by only taking drugs. By the assistance of insulin injection, the body can maintains the sugar circulating in the bloodstream within a healthy level.

Different to the internal factors, the impact trend of these external factors are similar to most people. And the blood glucose concentrations can be well modified by regulating these external factors.
