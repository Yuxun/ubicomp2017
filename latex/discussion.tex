\section{Discussion}
\label{sec:discussion}
\subsection{Usability}
\textcolor[rgb]{1.00,0.00,0.00}{This work focuses primarily on the challenges to monitor the blood glucose dynamics with few CGM calibration by the smartphones. A limitation of the existing prototype lies in collecting manually recorded data from users. As mentioned in section \ref{sec:user_study}, this issue can be handled with some current technologies such as computer vision \cite{bib:kawano2015foodcam}  and speech recognition \cite{bib:hinton2012deep}, and a camera API has been opened for users to take photos for their dishes, drug and insulin intake for the potential use. We are now optimizing the human-computing interaction on this side.}

\subsection{Inference error}
\textcolor[rgb]{1.00,0.00,0.00}{As mentioned in section~\ref{subsec:Inference_Results}, most errors come from two periods: 1) the transition of two blood glucose level 2) the duration of sudden blood glucose concentration change.
Based on the study of experiment, errors during blood glucose transitions are mainly brought by the temporary delays to measure the external factors. In this case, the inference result will follow up the true value in a short time once these external factors detected. Errors of sudden blood glucose fluctuation are often resulted by the abrupt or tiny changes of the external factors,
which may not be immediately detected by \sysname. In this case, the durations of errors are often too short to cause risky emergencies \cite{bib:Low_Blood_Glucose_(Hypoglycemia),bib:whitmer2009hypoglycemic}. In addition, 3.81\% Type D Clarke error in Table~\ref{predict_results} increases the false negative detection of blood glucose.
As mentioned in section~\ref{subsec:predict_result_analysis}, \sysname adopts a warning delay (9 mins) to copy with this issue. For the 0.91\% Type E error, it rarely happen and often occurs while the training and testing gap become large (more than 20 days). When the users find this case by double-check (finger sticker or CGM), we recommend users to wear CGM to collect data for periodic retraining. In the future, we will optimize the detection performance of \sysname to reduce these two types of errors.}


%Specifically, most errors belong to the results of Type B in \ref{subsec:predict_result_analysis}. }
