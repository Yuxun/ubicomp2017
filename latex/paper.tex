\documentclass[acmlarge]{acmart}

% Metadata Information
\acmJournal{IMWUT}
\acmVolume{2}
\acmNumber{0}
\acmArticle{0}
\acmYear{2017}
\acmMonth{6}
\acmArticleSeq{0}

% Copyright
\setcopyright{acmcopyright}
%\setcopyright{acmlicensed}
%\setcopyright{rightsretained}
%\setcopyright{usgov}
%\setcopyright{usgovmixed}
%\setcopyright{cagov}
%\setcopyright{cagovmixed}

\acmDOI{0000001.0000001}

% Paper history
\received{February 2017}
%\received[accepted]{June 2009}


% Load basic packages
%\usepackage{balance}       % to better equalize the last page
\usepackage{booktabs}
\usepackage{multirow}
\usepackage{colortbl}
%\usepackage[usenames, dvipsnames]{color}
%\usepackage{epstopdf}
%\usepackage[T1]{fontenc}   % for umlauts and other diaeresis
%\usepackage{graphics}      % for EPS, load graphicx instead
%\usepackage{hyperref}
%\usepackage{mathptmx}
\usepackage[caption=false,font=normalsize]{subfig}
%\usepackage{tabularx}
%\usepackage{textcomp}
%\usepackage{txfonts}
%\usepackage{url}
\usepackage{threeparttable}
\usepackage{xspace}
\usepackage{rotating}
\usepackage{array}


% End of preamble. Here it comes the document.

\newcommand{\etal}{\emph{et~al.}\xspace}
\newcommand{\eg}{\emph{e.g.},\xspace}
\newcommand{\ie}{\emph{i.e.},\xspace}
\newcommand{\etc}{etc.\xspace}
\newcommand{\cf}{cf.\/~}
\newcommand\figref[1]{Fig.~\ref{#1}}
\newcommand\figsubref[1]{Fig.~\subref{#1}}
\newcommand\tabref[1]{Table~\ref{#1}}
\newcommand\tabsubref[1]{Table~\subref{#1}}
\newcommand\secref[1]{Sec.~\ref{#1}}
\newcommand\equref[1]{(\ref{#1})}
\newcommand\appref[1]{(App.~\ref{#1})}

%\newcolumntype{C}{>{\centering\arraybackslash}X}

\newcommand{\fakeparagraph}[1]{\vspace{1mm}\noindent\textbf{#1.}}
\newcommand{\TODO}[1]{\textbf{\color{red}TODO:{ #1} }}

\newcommand{\zimu}[1]{{\color{Maroon}{#1}}}
\newcommand{\weixi}[1]{{\color{blue}{#1}}}

\newcommand{\sysname}{SugerMate\xspace}
\newcommand{\modelname}{Md$^3$RNN\xspace}


\newcommand{\subsecvspace}{\vspace{-0.15cm}}

\begin{document}
%\title{SugerMate: Towards Ubiquitous Smartphone-based Blood Glucose Inference}
\title{SugerMate: Non-intrusive Blood Glucose Inference with Smartphone}

\author{Weixi Gu}
\affiliation{%
  \institution{Tsinghua University}
  %\department{}
  %\streetaddress{ }
  %\city{ }
  %\postcode{ }
  %\country{ }
}
\author{Yuxun Zhou}
\affiliation{%
  \institution{University of California,Berkeley}
  %\department{}
  %\streetaddress{ }
  %\city{ }
  %\postcode{ }
  %\country{ }
}
\author{Zimu Zhou}
\affiliation{%
  \institution{ETH Zurich}
  %\department{}
  %\streetaddress{ }
  %\city{ }
  %\postcode{ }
  %\country{ }
}

\author{Costas J. Spanos}
\affiliation{%
  \institution{University of California,Berkeley}
  %\department{}
  %\streetaddress{ }
  %\city{ }
  %\postcode{ }
  %\country{ }
}

\author{Lin Zhang}
\affiliation{%
  \institution{Tsinghua University}
  %\department{}
  %\streetaddress{ }
  %\city{ }
  %\postcode{ }
  %\country{}
}

\begin{abstract}
  Inferring abnormal glucose events such as hyperglycemia and hypoglycemia is crucial for the health of both diabetic patients and non-diabetic people.
  However, continuous or regular blood glucose monitoring can be invasive and inconvenient in everyday life.
  In this paper, we present \sysname, a non-intrusive smartphone-based abnormal blood glucose monitor system.
  Provided with inputs of food, drug and insulin intake, it leverages smartphone sensors to automatically measure physical activities and sleep quality, and then infers the current blood glucose level at a fine-grained time resolution.
  Nevertheless, accurate blood glucose level inference is still challenging due to model learning difficulties brought by the imbalanced, personalized, and often limited measurements.
  To this end, we propose \modelname (multi-division deep dynamic recurrent neural network), a novel learning paradigm that is able to make full use of the available measurements.
  Specifically, \modelname captures complex, multi-scale blood glucose dynamics via deep networks, extracts grouped feature representations with a multi-division learning structure, and preserves user-specific characteristics using personalized output layers.
  Evaluations on 112 users over 7 months show that \modelname yields an average accuracy of 82.14\%, significantly outperforming previous learning methods that are either shallow, generically structured, or oblivious to grouped behaviors.
\end{abstract}
%
% The code below should be generated by the tool at
% http://dl.acm.org/ccs.cfm
% Please copy and paste the code instead of the example below.
%
\begin{CCSXML}
<ccs2012>
<concept>
<concept_id>10003120.10003138.10003140</concept_id>
<concept_desc>Human-centered computing~Ubiquitous and mobile computing systems and tools</concept_desc>
<concept_significance>500</concept_significance>
</concept>
<concept>
<concept_id>10003120.10003138.10003141.10010895</concept_id>
<concept_desc>Human-centered computing~Smartphones</concept_desc>
<concept_significance>500</concept_significance>
</concept>
<concept>
<concept_id>10003120.10003138.10003141.10010897</concept_id>
<concept_desc>Human-centered computing~Mobile phones</concept_desc>
<concept_significance>500</concept_significance>
</concept>
<concept>
<concept_id>10010147.10010178</concept_id>
<concept_desc>Computing methodologies~Artificial intelligence</concept_desc>
<concept_significance>500</concept_significance>
</concept>
<concept>
<concept_id>10010147.10010257</concept_id>
<concept_desc>Computing methodologies~Machine learning</concept_desc>
<concept_significance>500</concept_significance>
</concept>
</ccs2012>
\end{CCSXML}

\ccsdesc[500]{Human-centered computing~Ubiquitous and mobile computing systems and tools}
\ccsdesc[500]{Human-centered computing~Smartphones}
\ccsdesc[500]{Human-centered computing~Mobile phones}
\ccsdesc[500]{Computing methodologies~Artificial intelligence}
\ccsdesc[500]{Computing methodologies~Machine learning}
%
%%
%% End generated code
%%
%
%% We no longer use \terms command
%\terms{Design, Algorithms, Performance}
%
%\keywords{Wireless sensor networks, media access control,
%multi-channel, radio interference, time synchronization}


\thanks{
Authors address: Weixi Gu and Lin Zhang, Tsinghua-Berkeley Shenzhen Institute, Shenzhen, China;
E-mails: guweixigavin@gmail.com, linzhang@tsinghua.edu.cn;
Authors address: Yuxun Zhou and Costas J. Spanos, Electrical Engineering and Computer Sciences, University of California, Berkeley, America;
E-mails: {yxzhou, spanos}@berkeley.edu;
Authors address: Zimu Zhou, Computer Engineering and Networks Laboratory, ETH Zurich, Switzerland;
E-mail: zzhou@tik.ee.ethz.ch;
}

\maketitle

% The default list of authors is too long for headers}
%\renewcommand{\shortauthors}{G. Zhou et al.}

% !TEX root = paper.tex

\section{Introduction}
\label{sec:intro}
Blood glucose concentration plays an important role in personal health.
Hyperglycemia (high blood glucose level) results in diabetes, leading to health risks such as pancreatic function failure, immunity reduce and ocular fundus diseases~\cite{bib:DC95:Klein}.
Meanwhile, hypoglycemia (low blood glucose level) also brings complications such as confusion, shakiness, anxiety, and if not treated in time, coma or death~\cite{bib:AJM91:DCCT}.
According to the World Health Organization, there are approximately 171 million people in the word suffering from diabetic patients.
The number of diabetic patients is expected to increase by more than 100\% by the year 2030 \cite{worldhealth}.



%Hypoglycemia characterizes a state of low glucose level in the bloodstream.
%While for non-diabetic people this state is relatively rare due to adequate regulation of the glucose level, it can lead to life threatening effects for diabetic patients, ranging from headaches to judgment impairment and loss of consciousness
%
%To help diabetic users regulate their glucose level, the standard method consists in collecting a drop of blood from the finger and analyze its glucose level using a glucometer.
%While this method is reliable as it is performed through a direct measurement, it is not very convenient as it requires the user to pinch
%her finger for each new observation.
%Furthermore, this method does not allow for a continuous monitoring, but rather a sporadic sampling of the glucose level.
%Alternatively, continuous glucose monitoring can be achieved using an underthe-skin sensor which relays glucose information to an electronic receiver.
%This method has a granularity of a sample every few minutes.
%However, the position of the sensor makes it cumbersome for an extended usage, limiting its applicability.
%
%In their research, Zecchin et al. [40] gathered data on control patients (N=20) and T1DM patients (N =19) over four days.
%For control and diabetic patients, the authors showed that moderate physical activity, corresponding to a daily activity, was associated with changes in glucose level, estimated by the first-and second-order glucose concentration time derivatives.
%
%Physical activity was also found to significantly reduce postprandial glucose excursions in T1DM and healthy participants [24].

%However, there is not currently a consensus about how to include physical activity and other intra-patient variability sources in the glucose kinetics models.

Current continuous glucose monitors (CGM) either rely on the electrochemistry \cite{wang2008electrochemical, chen2013recent} or light reflection \cite{van1987blood} to track the variance of blood glucose.
Both of them, however, are usually limited to clinical uses.
Complicated operations and physical intrusive requirement make them hard to be accepted by the public.
Accordingly, it is not convenient for people to wear this kind of products at all times.


To this end, we propose \sysname, a non-intrusive and pervasive mobile service for abnormal blood glucose track in daily use.
The key insights of \sysname are based on the following aspects.
On one hand, the blood glucose level is impacted by the outer contextual factors, including the physical activities (\eg running and walking), intakes of food and clinical drugs, time and user's sleep quality \cite{zainuddin2009neural}.
Such factors can be detected via off-the-shelf smartphones.
On the other hand, the blood glucose is also determined by the physical genetic factor.
The genetic factor is usually different between crowds yet same in each person or similar within a group of people~\cite{lynn2002variation}.
Accordingly, \sysname models the blood glucose trend of each user based on the historical blood glucose values while he/she is not wearing the CGM.
When \sysname detects the abnormal points of blood glucose, it reminds the user of measuring his/her blood glucose values by the clinical CGM for a double-check.
By combining the clinical CGM and \sysname, users are able to acknowledge their blood glucose variance at any moment.

The implementation of \sysname is very challenging.

First, the blood glucose measurement of the professional CGM device is for a short-term usage.
The glucose oxidase stored in the CGM device usually cannot supports to the blood measurement more than 4 to 5 days.
Moreover, most of the users are not willing to wear CGM devices again within a short period because of pains.
The limited mount of data makes the ad-hoc personal blood glucose analysis hard.
Second, the high or low blood glucose level is relative less than the normal level, resulting in the weak ability of \sysname to recognize these two levels. However, monitoring the high and low blood levels is of significant importance,  how to promise their detection accuracy is great essential.
Third, the traditional physiological model of tracking the blood glucose variance \cite{bib:phy} cannot be applied to \sysname directly. It assumes that the impact of external factors (\eg food intake, calories cost of exercises, drug and insulin intake) are same to all the people, but ignores the individual differences. Meanwhile, the prediction intervals of \sysname is inconsistent with that of the traditional physiological model.
The unified parameters of physiological, therefore, cannot well handle with the blood glucose variance of different individuals and different predict time.

To address the aforementioned issues, we track the blood glucose of a person  through a data-driven perspective, by  building up a multi-task deep RNN model to merge the blood glucose data of persons but still guarantee its characteristics of each single person.

%Those who share the similar diabetes conditions are clustered according to the basic information (\emph{e.g.} age, gender, weight, diabetes types and the year of diagnose).  We conduct feature representation learning on each group in the first layer of the multi-task model. Considering the temporal dynamics are

The key contributions of our work are listed as follows:

\begin{itemize}
  \item
  A non-intrusive and ubiquitous approach is proposed to monitor the blood glucose variance with smartphones.
  It monitors the user's blood glucose level when he/she does not wear clinical professional monitor devices.
  Once \sysname detects the abnormal points of blood glucose level, it reminds the user of  blood glucose measurement by CGM for a further control.
  \item��
  Two dimensional types of blood glucose features (\emph{e.g.},  physiological factors  and  temporal factors) are well considered to infer the blood glucose levels. In particular, the physiological factors described in the physiological model are encoded in the Multi-task deep RNN model, using for quantifying the impact of physiological factors on the blood glucose levels. We also translate the historical data to infer the current blood glucose level. By measuring the influence of these factors on the blood glucose, \sysname can well infer the variance of the blood glucose.
  \item
  By sharing the blood glucose data in the information representation layer and temporal dynamic deep learning layer, \sysname can well copy with the limited data of single user, but enables to keep the individual blood glucose characteristics in the personality learning layer.
  Meanwhile, the deep Recurrent Neural Networks (RNN) adopted in \sysname, is able to encode the temporal relationships between the sensed outer contextual factors and blood glucose level.
  With the assistance of the multi-task deep RNN model, \sysname can infer the blood glucose trend with high accuracy.
\end{itemize}


% !TEX root = paper.tex
\section{Related Work}
\label{sec:relwork}
Research on inferring blood glucose concentrations or abnormal events dates back to the 1960s and continues to attract extensive research interest~\cite{bib:IJNMBE16:Oviedo}.
Physiological models~\cite{bib:TBE07:Dalla, bib:PE04:Hovorka} mathematically formulate the whole process of glucose metabolism and are widely used for simulations and studies involving glucose regulation.
One major drawback of physiological models is the requirement for prior knowledge to adjust the physiological parameters.
Alternatively, researchers propose to combine machine learning techniques with a non-specific physiological model or directly correlating blood glucose levels with insulin, food intake and other inputs without physiological parameters.
For instance, Plis~\etal~\cite{bib:MAIHA14:Plis} apply a generic physiological model of blood glucose dynamics to extract features for support vector regression to infer blood glucose levels.
Reymann~\etal~\cite{bib:EMBC16:Reymann} replace the physiological model by an online simulator and bring blood glucose tracking on mobile platforms.

While physiological models and the underlying glucose metabolism dominate the dynamics of blood glucose, the impact of seemingly ``secondary'' factors, such as those related with individual's lifestyle, can be quite significant.
Variations of the personalized external lifestyle factors such as meals, insulin or drug intake, exercises, and sleep quality, \etc, can also lead to blood glucose changes, which are not captured by a universal physiological model~\cite{bib:DRCP15:Iwasaki}.
Consequently, it is crucial to monitor these external lifestyle factors as inputs to improve the performance of blood glucose concentration inference.
METABO~\cite{bib:EMBC09:Georga} is a client-server architecture based system that records dietary, physical activity, medication and medical information for hypoglycaemic and hyperglycaemic event prediction.
Marling~\etal~\cite{bib:KDHealth16:Marling} improve hypoglycemia detection by combining CGM data with heart rate, galvanic skin response and skin temperature collected from a fitness band.
However, these works all require CGM data as input, making them invasive and inconvenient for both patients and non-diabetic people.
\textcolor[rgb]{1.00,0.00,0.00}{Compared with previous works, our system requires less calibration of CGM. It only necessitates per suer calibration data from a CGM at the training period, but does not rely on CGM input at the testing period.}

Alternatively, there has been attempt at non-invasive blood glucose monitoring with pervasive wearable and mobile devices.
Nguyen~\etal~\cite{bib:EMBC12:Nguyen} observe distinct patterns in ECG signals during hypoglycemia and hyperglycemia in type I diabetic patients.
Sobel~\etal~\cite{bib:JDST14:Sobel} integrate five types of sensory data from two accelerometers, a heat-flux sensor, a thermistor, two ECG electrodes and a galvanic skin response sensor to predict blood glucose concentration.
Ranvier~\etal~\cite{bib:SEMPER16:Ranvier} leverage ECG signals, and energy expenditure (estimated by an accelerometer and a breathing sensor) to detect hypoglycemic events.
Our work is inspired by this body of research. We propose a smartphone-based non-invasive blood glucose monitoring system that jointly considers meals, drugs and insulin intake, physical activity and sleep quality without CGM data as inputs. \textcolor[rgb]{1.00,0.00,0.00}{The main usability advantage lies in the less CGM calibration as input. }
Practically, physical activity level and sleep quality are automatically tracked without manual input, which notably improves the useability of our system.


Moreover, personalized blood glucose models are also important. It is because those models with generic parameters may not reflect user-specific factors, such as age, weight and insulin-to-carbohydrates ratio~\cite{bib:IJNMBE16:Oviedo}. 
Both the physiological parameters and the impact of life events on blood glucose need to be trained on user-specific data to account for inter-person differences~\cite{bib:ICMLA13:Bunescu}.  Several classic statistic models can be used to train these parameters by the machine learning algorithms.
\textcolor[rgb]{1.00,0.00,0.00}{Support Vector Machines (SVMs)~\cite{bib:wang2005support} are the basic classification model.  They bases on the idea of optimal separating hyperplane that maximizes the separation margin of two data groups (classes). Due to this construction, it usually generalizes well, and its dual form is a quadratic programming that can be easily incorporated with kernels.
Gaussian Process~\cite{bib:rasmussen2006gaussian} is a particular kind of statistical model where observations occur in a continuous domain, e.g., time or space. In a Gaussian process, every point in some continuous input space is associated with a normally distributed random variable.
Gradient Boosting~\cite{bib:friedman2002stochastic} generates a prediction model by combining many weak classifiers into a stronger classification committee.
It is an accurate and effective off-the-shelf procedure, so it is often used in a variety of areas such as web search ranking and click through rate prediction. However, these methods have no sequential structure, difficulty in capturing the temporal dependency of the blood glucose data.
Hidden Markov Models (HMMs)~\cite{bib:rabiner1986introduction} are statistical Markov models where the system being modeled are assumed to be a Markov process with unobserved states. It has been widely used in signal processing and time series analysis due to its interpretability and tractability, such as the application in speech recognition~\cite{bib:gales2008application}.
Recurrent Neural Network (RNN)~\cite{bib:pearlmutter1989learning} is an effective approach to sequential prediction. Distinguished with the feedforward neural network, RNN can use their memory information to process sequences of inputs. It brings much improvement in a wide variety tasks such as handwriting recognition~\cite{bib:graves2008unconstrained} and speech recognition~\cite{bib:graves2013speech}.}
Nevertheless, a primary impediment to build up such models is lacking of sufficient personalized blood glucose data ~\cite{bib:KDHealth16:Marling}, easily resulting in frustrated classification results.

\textcolor[rgb]{1.00,0.00,0.00}{In this paper, we advance previous works by carefully designing a multi-division RNN framework. It shares blood glucose information among groups of users, but preserves user-specific blood glucose characteristics via personalized learning, thus making full use of the limited, sometimes incomplete user-specific data, and achieving higher prediction accuracy than both generic learning and personalized learning. Shared inputs are a hallmark of deep learning architectures. Many successful RNN architectures use a shared feature extraction method (\eg, auto encoding) as input to a personalized output \cite{bib:lane2015deepear}. Compared with these works, \modelname not only shares personalities, but also encodes grouped and embedded features into a share layer, comprehensively modelling the characteristics of different diabetic types.  }


%Useful:
%Meals, physical activity, and emotional state are some of the factors that affect BGC.
%Using information from such factors improves the performance of BGC regulation.
%Physical activity information was used in hypoglycemia alarm system [21], [25] and in control system [15], [16], [26] to predict and prevent low BGC after physical activity.
%Meal information is used by many researchers to compute the amount of insulin bolus to be infused.
%However, use of information manually entered by patients should be balanced with convenience and adherence.
%Patients may forget to enter meal information in a timely manner or make erroneous estimates about the carbohydrate content of the meal.
%The protein, fat, and carbohydrate ratios of the foods impact the glycemic value of the meal ingested.
%The proposed modeling and control algorithms do not require any announcements from the patient.
%Simulations based on the UVa simulator [8] use only continuous glucose monitor (CGM) readings.
%Control studies in clinical experiments use CGM readings and physiological information from a sports armband [27].
%The simulation and experimental results reported illustrate that good control is achieved in both cases with no hypoglycemia episodes.


%
\section{Preliminary}
\label{sec:preliminary}
This section briefly reviews blood glucose levels and the impacting factors.

\subsection{Blood Glucose Levels}
The body of a healthy individual regulates the blood glucose concentration within 4.4 mmol/L to 6.1 mmol/L~\cite{bib:BGWiKi}.
The blood glucose can grow slightly to 7.8 mmol/L after eating and usually returns to the normal range afterwards.
Persons with the blood glucose concentration above 7.8 mmol/L (hyperglycemia) for a prolonged period are at the risk of diabetes mellitus, and need hypoglycemic drugs or insulin injection.
In contrast, if the blood glucose concentration drops to below 4.4 mmol/L, it is a sign of hypoglycemia.
Since we are interested in detecting normal and abnormal blood glucose events, we divide blood glucose concentration into 4 blood glucose levels as shown in \tabref{tab:blood_glucose_levels}.
%Glucose is transported from the intestines or liver to body cells via the bloodstream, and is made available for cell digestion via the hormone insulin, released by the pancreas. 
%While the glucose level rising, the pancreas produces insulin to promote the transfer of glucose from the blood to the cells of muscle and fat, and to increase the uptake of glucose into the liver. 
%While the glucose level decreasing, the insulin level drops as well and the liver increases glucose production to bring the blood glucose level back to normal.

\begin{table}[h]
  \centering
  \caption{Normal and abnormal blood glucose levels.}
  \label{tab:blood_glucose_levels}
  \begin{tabular}{|c|c|c|}
  \hline
  \textbf{Blood Glucose Value (mmol/L)} & \textbf{Glocose Level} & \textbf{Explanation}                      \\ \hline
  (0, 4.4{]}                            & Level 1                & Low Blood Glucose                          \\ \hline
  (4.4, 6.1{]}                          & Level 2                & Normal Level of Fasting Blood Glucose      \\ \hline
  (6.1, 7.8{]}                          & Level 3                & Normal Level of Postprandial Blood Glucose \\ \hline
  (7.8, $+\infty$)                               & Level 4                & High Blood Glucose                         \\ \hline
  \end{tabular}
\end{table}

\subsection{Impact Factors of Blood Glucose}
The complete physiology of blood glucose is complex and it is believed that the blood glucose concentration is affected by both \emph{internal} and \emph{external} factors.
Internal factors include the self-management of blood glucose, which is specific for each individual.
External factors include activities that directly input or catabolize blood glucose such as food, drug, or insulin intake, and physical activities such as exercises.
%Blood glucose level is mainly determined by two dimensional factors:  the internal factors and the external factors.
%The internal factors indicates the genetics of persons.
%Many clinic researches or reports have demonstrated genetics are associated with diabetes \cite{bib:simpson1978genetics,bib:diabetes_co_uk,bib:rotter1984genetics,bib:concannon2009genetics,bib:salopuro2004common}.
%It is different from individuals.
The external factors are mainly composed of food intakes, the exercise, the sleep quality and drugs or insulin inputs~\cite{bib:duke2010intelligent}. 
The physiological mechanism of external factors on blood glucose are shown as follows.

\paragraph{Food intake} 
The carbohydrates of the food intake can be quickly absorbed by gut and turned into the blood glucose. 
A steep rising trend usually occurs after consuming high-carbohydrate meals. 
Naturally, high-carbohydrate foods (\eg white grain products, bread and cookies) contributes higher blood glucose levels than low/moderate-carbohydrate foods (\eg meat, vegetables).
Hence, the food intake is an important external factor of blood glucose.

\paragraph{Exercise} 
Exercise impacts the blood glucose level by the energy consumption. 
The movement of muscles will trigger the body cells absorbing sugar from blood for energy of doing exercise.  
By doing exercises, physical activity can help lower the blood glucose level for several hours after stop moving. 
The regular exercises promote the sensitivity of cells to the insulin, which can help keep blood glucose level vary in a normal range.

\paragraph{Sleep Quality} 
Sleep quality is strongly linked to the blood glucose levels. 
Several clinical studies \cite{bib:scheen1998roles,bib:scheen1996relationships,bib:spiegel2005sleep} have demonstrates poor sleep quality is likely to results in high glucose levels. 
The body's reaction to sleep loss can resemble insulin resistance, a precursor to diabetes. 
Under this case, the cells of body easily fail to generate hormone efficiently, resulting in high blood glucose therefore. 
Moreover, sleep disorder is a key culprit of overweight \cite{bib:punjabi2002sleep, bib:knutson2006role}, which is also closed associated with the diabetes.

\paragraph{hypoglycemic drugs} 
The hypoglycemic drugs are mainly used for the patients with type II diabetes\cite{bib:eurich2007benefits,bib:jung2006antidiabetic,bib:patel2012overview}. 
Usually, the anti-diabetes medication control the blood glucose concentrations by three ways: (1) stimulating the insulin secretion from pancreas, (2) promoting the tissue sensitivity to insulin, (3) low the rate at which glucose is absorbed from the gastrointestinal tract.

\paragraph{Insulin} 
Insulin injection is a common way to treat the diabetes. 
For those who suffer Type I diabetes, they have to inject insulin every day due to their bodies fail to release insulin. 
For those who have Type II diabetes, insulin injection is also required when your body cannot proper react to insulin or produce enough of it by only taking drugs. 
By the assistance of insulin injection, the body can maintains the sugar circulating in the bloodstream within a healthy level.

Different to the internal factors, the impact trend of these external factors are similar to most people. 
And the blood glucose concentrations can be well modified by regulating these external factors.


\section{Overview}
\label{sec:overview}

\begin{figure}[!t]
  \centering
  \includegraphics[width=1\columnwidth]{./img/System_Arch.pdf}
  \caption{The system architecture}
  \label{fig:architecture}
\end{figure}



The framework of \sysname is composed of three major components: external factor collection module, multi-task deep RNN module and blood glucose level tracking module. 

In the external factors collection module, the users are required to recrd their information into application, including of the diabetes type, drug, insulin and food intake. Meanwhile, \sysname triggers the embedded sensors to sense the physical activities and sleep quality of user.
\sysname classifies the users into three groups based on glucose types afterwards.
In the multi-task deep RNN (\modelname) model, the feature representation is firstly learnt within the users in a same group. Then, a deep RNN layer is trained based on the dataset of all users, which is to establish the dynamic relationships between the outer contextual factors and the corresponding blood glucose level. In the last, \modelname learns the personality of each user by the personality layer.
Based on the results of \modelname model, \sysname tracks the current blood glucose level in the last module. Once \sysname detects the abnormal points of blood glucose ( \ie in a \emph{high} or \emph{low} blood glucose level ), it reminds the user to measure the blood glucose by a clinical CGM or finger pricking method for a double-check.






%The framework of \sysname is shown as \figref{fig:architecture}, consisting of four major components.
%The first one is the external factors collection.
%The users are required to enter their basic information into application, including of the age, the gender, the diabetes type and the year of diagnosis.

%After a user measures his/her blood glucose by a CGM, the records of the blood glucose along with the outer contextual factors occurred during the measurement are uploaded to an individual database automatically.
%Afterwards, a RNN model is trained based on the dataset to establish the relationships between the outer contextual factors and the corresponding blood glucose level.
%It then is fed into the user's smartphone.
%When the user does not wear the CGM, \sysname detects the outer contextual factors with embedded sensors in the smartphones, and infers the current blood glucose level based on the trained model.
%Once \sysname detects the abnormal points of blood glucose ( \ie in a \emph{high} or \emph{low} blood glucose level ), it reminds the user to measure the blood glucose by a clinical CGM for a further control.

The extraction mechanisms of outer contextual factors are detailed as follows.

\emph{Physical activity}:
\sysname leverages the accelerometer to detect the user's activities by the approaches in \cite{bayat2014study}, as well as the corresponding time costs.
\sysname then measures the calorie of user's physical consumption.

\emph{Food intake}:
\sysname measures the food's effect on a person's blood glucose level based on the glycemic index.

\emph{Clinical drug intake}:
\sysname records the name and amount of the drug that user eat.

\emph{Time}:
\sysname invokes the timer embedded in the smartphone to record the time.

\emph{Sleep quality}:
\sysname measures the user's sleep by the approach in \cite{gu2014intelligent}.


\section{Design}
\label{sec:design}

\subsection{External factor collection}
\label{subsec:external}
Sensing module is mainly designed for collected the external factors of users.

\paragraph{Food intake}
As the food intake is a main source of carbohydrate, \sysname provides the food menu for users to record their daily intakes based on the carbohydrate food list \cite{bib:carblist}.
Five common food categories have been provided by \sysname, including grains ,vegetables, mike and egg,fruits and meats. The users are asked to enter their food items and the corresponding amounts. \sysname calculate the carbohydrate of a meal and measure its impact on blood glucose level.

\paragraph{Drug intake}
The oral diabetes medications enhance secretion of insulin into the blood by the pancreas or decrease amount of glucose released from liver, keeping the blood glucose in a low level for type II diabetes.
In \sysname, a drug menu of 11 oral diabetes is report for users to input their drug intake. After eating the diabetes drugs, users select their pills name and report the drug dosage. The drug list is provided based on \cite{bib:druglist}.
\sysname transfers the drug dose as the blood glucose efforts according to their work functions \cite{bib:druglist, bib:bolen2007systematic, bib:bennett2011oral}  by physiological model.

\paragraph{Insulin injection}
Insulin injection is to control blood glucose concentration of those who have type I diabetes, and the patients of type II, whose blood sugar is too high for their bodies to control. \sysname provides a insulin type list based on \cite{bib:insulinlist} for user with diabetes to enter their usage and insulin dosage, and then transfer it into physiological model to compute the blood glucose level.

\paragraph{Activity factors}
Since the carbohydrate in the body can be consumed by daily exercise, resulting to varying the blood glucose level, \sysname adopts an effective and power efficient approach \cite{bib:kwapisz2011activity} to automatically recognize six common user's daily activities (i.e., walking, running, upstairs, downstairs, sitting and standing), and record the corresponding durations. The caloric expenditure can be easily consumed by the calorie burn calculator formulas as Equation ~\ref{calorie_burn}.
\begin{equation}\label{calorie_burn}
  Calorie Burn = (BMR/24)*MET*T,
\end{equation}
where BMR (Basal Methobolic Rate) is the amount of energy required to simply sit or lie quietly \cite{bib:mcnab1997utility}, and MET (Metabolic Equivalent) is the ratio of the work metabolic rate to the resting metabolic rate\cite{bib:ainsworth2000compendium}. The calculator formula has been widely used by multiple sport applications \cite{bib:shapesense, bib:HealthStatus, bib:CalorieCounter}.
\sysname finally leverages the calories to measure the effects of exercise on the blood glucose.

\paragraph{Sleep quality}
Sleep quality has a long influence on blood glucose level. To measure the sleep quality of users, \sysname applies an effective method in \cite{bib:gu2014intelligent} to measure the sleep quality, and leverages the sleep quality score as an index to evaluate the sleeping impact on blood glucose level.

\subsection{Feature Engineering of Physiological-Temporal Views }
\subsubsection{Physiological View}

\begin{figure}[t]
  \centering
  \includegraphics[width=0.9\columnwidth]{./img/Physiological_correlation1.pdf}
  \caption{Temporal graph of the physiological model.}
  \label{fig:phymodel}
\end{figure}
The feature engineering of physiological view is to leverage a physiological model to quantify the dynamics of physiological factors in the body. The
physiological model, based on the physiology mechanism of blood glucose in Section~\ref{sec:preliminary}, has been widely studied in previous works \cite{bib:briegel2002nonlinear,bib:duke2010intelligent, bib:plis2014machine}.It primarily measures the real-time values of carbohydrate, insulin and glucose influenced by the external factors. We constructed our physiological model based on the work \cite{bib:duke2010intelligent},
with an extension of sleeping fact according to its physiological impact discussed in Section~\ref{sec:preliminary}.

\paragraph{Temporal Graph of Physiological Model}
The physiological model of \sysname describes the physiological factors from five aspects: carbohydrate dynamics, insulin dynamics, exercise dynamics,
sleep dynamics and blood glucose dynamics.
%Given the hidden physiological vectors $X_{t}=\{C_{g1}(t), C_{g2}(t),\\ I_{s}(t),I_{m}(t), I_{t}, G_{m}(t), G(t), E(t)\}$, where the elements of this vector represent
%the hidden physiological factors at time point $t$, and the observable input vector $U_{t}=\{U_{c}(t), U_{e}(t), U_{s}(t), U_{i}(t)\}$ , in which $U_{c}$ stands for the
%carbohydrate proportion of meals, $U_{e}$ indicates the calories cost by exercise,  $U_{s}$ is the sleep score and $U_{i}$ states for the amount of insulin injected or simulated
%by the diabetes drugs  at time point $t$. In particular, the sleep quality $U(s)$ is a constant during a whole day.
%The station transition functions can be represented as $X_{t+1}=f(X_t, U_t)$, and its corresponding
%temporal transition graph is shown as \figref{fig:phymodel}.  %\tabref{phy_tab} details the transformation functions of the carbohydrate $C_{g1}$ and $C_{g2}$, and insulin
%As \figref{fig:phymodel} shown, the equations of $C_{g1}$ (Equation~\ref{Eq:Cg1}) and $C_{g2}$ (Equation~\ref{Eq:Cg2}) indicating the temporal transitions of carbohydrate consumption and the carbohydrate digestion respectively.

%Positive factors of physiological model indicate the carbohydrate absorption and the hepatic glucose production, which will increase the blood glucose level.
%Negative factors of  physiological model describe the insulin independent uptake, insulin dependent uptake and the renal clearance. We will detail these factors as follows:


\emph{Carbohydrate dynamics}: Carbohydrate dynamics refers to the transitions of carbohydrate consumption  $C_{g1}$ and the carbohydrate digestion $C_{g2}$. 
Equation~\ref{Eq:Cg1}  and Equation~\ref{Eq:Cg2} show their transition equations respectively, where $U_{c}$ stands for the carbohydrate proportion of meals.
\begin{equation}\label{Eq:Cg1}
C_{g1}(t+1)=C_{g1}(t)-\alpha_{1}^c*C_{g1}(t)+U_{c}(t)
\end{equation}

\begin{equation}\label{Eq:Cg2}
C_{g2}(t+1)=C_{g2}(t)+\alpha_{1}^c*C_{g1}(t)-\alpha_{2}^c*C_{g1}(t)
\end{equation}

\emph{Insulin dynamics}: Insulin dynamics indicates the transitions of subcutaneous insulin $I_{s}$ (Equation~\ref{Eq:Is}), 
the insulin mass $I_{m}$ (Equation~\ref{Eq:Im}),  and the level of active plasma insulin $I$ (Equation~\ref{Eq:I}). $U_{I}$ 
states for the amount of insulin injected or simulated by the diabetes drugs. $S^I$ and $bm$ refer to the insulin sensitive 
and body mass respectively.

\begin{equation}\label{Eq:Is}
I_{s}(t+1)=I_{s}(t)-\alpha_{f}^I*I_{s}(t)+U_{I}(t)
\end{equation}


\begin{equation}\label{Eq:Im}
I_{m}(t+1)=I_{m}(t)-\alpha_{f}^I*I_{s}(t)-\alpha_c^I*I_{m}(t)
\end{equation}

\begin{equation}\label{Eq:I}
I(t)=\frac{I_{m}(t)*S^I}{142*bm}
\end{equation}

\emph{Exercise dynamics}: Exercise dynamics $E$ denotes the exercise effect on insulin over the past time window. This 
long-term influence can be expressed by a cumulative moving average \cite{bib:lowry1992multivariate, bib:cma} as Equation~\ref{Eq:E} in the physiological model.

\begin{equation}\label{Eq:E}
E(t-t_0+1)=(t-t_0)*E(t-t_0)+U_{e}(t-t_0)
\end{equation}

where $t$ and $t_0$ are the current and beginning time point in the past time window. In \sysname,
the window size of exercise is set to 24 hours, which optimizes the experimental results and matches the conclusion
of clinical studies.

\emph{Sleep dynamics}: Sleep dynamics $S$ represents the sleeping quality effect on insulin. In physiological model, sleeping
effect also has a long-term influence on the insulin as the exercise effects. Specifically, it maintains a constant effect on 
blood glucose for each day. Equation~\ref{Eq:S} shows its transition equation.

\begin{equation}\label{Eq:S}
S(t-t_0+1)=(t-t_0)*S(t-t_0)+U_{s}(t-t_0)
\end{equation},

where $t$ and $t_0$ are the current and beginning time point in the past time window. 
In \sysname, the window size of sleep lasts for 7 days, which optimizes the experimental results and matches the conclusion
of clinical studies.


Insulin dynamics indicates the transitions of subcutaneous insulin $I_{s}$ (Equation~\ref{Eq:Is}),
the insulin mass $I_{m}$ (Equation~\ref{Eq:Im}),  and the level of active plasma insulin $I$ (Equation~\ref{Eq:I}). $U_{I}$
states for the amount of insulin injected or simulated by the diabetes drugs. $S^I$ and $bm$ refer to the insulin sensitive
and body mass respectively.


The impacted results $\delta_{abs}$ of carbohydrate absorption increase the blood glucose concentration can be described as 
$\delta_{abs}=\frac{\alpha_3^c*\alpha_2^c}{1+25/C_{g2}}$

Hepatic glucose production refers to the glucose released by the liver. It depends on the current blood glucose concentration G(t) 
and level of plasma insulin I(t).


The equations of $I_{s}$ (Equation~\ref{Eq:Is}) and $I_{m}$ (Equation~\ref{Eq:Im}) showing the temporal transitions of  the subcutaneous insulin
and the insulin mass respectively. $I$, the level of active plasma insulin, can be achieved by Equation~\ref{Eq:I}, where $S^I$ and $bm$ refer to the
insulin sensitive and body mass respectively.





Based on the positive factors and the negative factors, the blood glucose mass of individual body
$G_m$ can be calculated as Equation~\ref{Eq:Gm}
\begin{equation}\label{Eq:Gm}
G_m(t+1)=G_m(t)+\delta_{abs}-\delta_{ind}-\delta_{dep}-\delta_{clr}+\delta_{egp}
\end{equation},
where $\delta_{abs}=\frac{\alpha_3^c*\alpha_2^c}{1+25/C_{g2}}$

and the blood glucose concentration $G$ can be computed in Equation~\ref{Eq:G}.
\begin{equation}\label{Eq:G}
G_m(t+1)=G_m(t)+\delta_{abs}-\delta_{ind}-\delta_{dep}-\delta_{clr}+\delta_{egp}
\end{equation},


1) Positive factors:
Positive factors indicates the

However, the parameters of this model are different to individuals and also hard to be tuned.

\subsection{Blood glucose level prediction}

% !TEX root = paper.tex

\subsection{Intuition}
Given the features extracted from the physiological process, it seems plausible to perform any classification algorithm for blood glucose level prediction. Nonetheless, this plug-and-play approach will neglect important information from (1) dynamics of the process, and (2) inter-user similarity among same group of participants. Traditionally, various sequential classification methods[], e.g., hidden Markov model (HMM), recurrent neural networks (RNN), and dynamic conditional random fields (CRF), are used to capture the temporal correlation of the input feature. The inter process correlations are often times incorporated with the so-called multi-task learning approaches[], which learns processes (or tasks) in parallel to improve classification or to reduce the data sample requirement. 

In this paper, a novel machine learning paradigm, namely Multi-division deep-dynamic RNN (Md$^3$RNN), is proposed. To include the the aforementioned information sources in an unified framework, we develop two key ideas that extend the classical RNN. Firstly, the single hidden layer in RNN is replaced with several deep stacked layers. The deep structure in the new model is able to describe complex, mutli-scale system dynamics that would otherwise be ignored (or averaged out) by prior ``shallow'' models such as HMM, RNN, and CRF. Secondly, the correlations among users, being quite significant within user groups (divisions), are encoded by group-shared input layer and common hidden layers, whereas the distinct characteristics of individual users are modeled with  different output layers for personalized prediction. Within a larger scope of machine learning, the proposed Md$^3$RNN aims to leverage recent advancement of deep learning and multi-task learning, to model group-interacted time series data having complex temporal dynamics. It can be viewed as both a deep extension of RNN, and an intermediate between single-task learning and multi-task learning, hence the name Md$^3$RNN.

The overall configuration of the proposed model is summarized in Fig.\ref{fig:rnn}. Detailed construction of each component is given in the sequel.     

\begin{figure}[!t]
  \centering
  \includegraphics[width=0.9\columnwidth]{./img/pics_RNN.pdf}
  \caption{The Md$^3$RNN structure}
  \label{fig:rnn}
\end{figure}

\subsection{Model construction by layers}
The input of the Md$^3$RNN are the features extracted from the physiological model. The labeled data sequences for user number $j$ at time $t$ are denoted by $(x_i^{j},y_i^{j})$. We also adopt an index set convention, that $(x_A^{B},y_A^B)$ represents the data set $\left\{(x_i^{j},y_i^{j}) | i \in A, j\in B\right\}$ given index sets $A$ and $B$.
\subsubsection{Grouped Input Layer}
In the context of blood glucose prediction, available inputs are naturally divided into three groups according to the health condition of the participant from whom the data was generated. Notation-wise, we utilize $H$, $I$ and $II$ to indicate the the group of healthy user, user with type I diabetes and those having type II diabetes, respectively. Since the extracted features are essentially physiological indexes of an ``average'' person, they must go though different transformations to represent useful information of three distinct groups. This consideration motivate the design of the input layer (bottom of Fig.\ref{fig:rnn}) - it is divided into three units that performs different linear and non-linear transformation according to user groups. For instant, a data sample $x_t^{I_j}$, generated at time $t$ from the $j^{th}$ user of type I, undergoes the following processing:
\begin{align}
\tilde{x}_t^{I_j} = \sigma \left( W^Ix_t^{I_j} \right)
\end{align}     
where $W^I$ is the coefficients of the affine transformation \footnote{We assume that the interception is included in $W$. This can be done by simply adding a feature of all $1$s.}, $\sigma$ is some activation function, and $\tilde{x}_t^{I_j}$ is the output of the input layer for that data sample. Similar operations are conducted for data samples from group $H$ and $II$, but with different transformation coefficients. Intuitively, the shared transformation within groups would improve the learning of parameters (vs. single task learning), as information from all data in a homogeneous group is used.  

\subsubsection{Deep Dynamic Layer}
A common hidden layer is designated to capture the dynamics of the blood glucose evolution process. The underlying assumption is that, the biological and chemical reactions governing blood glucose variation are similar for all people, despite of grouped behaviors in the representation of physiological indexes (input layer), or individual characteristics in exhibited glucose level. This assumption could be justified by a series of medical research[][]. Moreover, since all users share the same hidden layer, all collected data samples are eventually helping the estimation of its parameters. The availability of rich information for the hidden layer makes the learning of a deep structure possible. In Md$^3$RNN, a number of Long Short Term Memory (LSTM) networks are stacked together (middle of Fig.\ref{fig:rnn}), to increase the overall model flexibility. In particular, it has been justified in both theory and practice that stacked LSTMs are able to capture dynamics occurring at different time scales, which in the current application would enable the modeling of both slow and rapid biological/chemical reactions. Mathematically, given the output from the grouped input layer, the deep dynamic layer performs
\begin{align}
h^0_t=
\end{align}



\subsubsection{Personalized Output Layer}


\subsection{Cost Sensitive Learning and Hyperparamter selection}

\section{Evaluation}
\label{sec:eval}

\subsection{Experimental Settings}
\begin{figure}[h]
  \centering
  \includegraphics[width=0.7\columnwidth]{./img/UI.pdf}
  \caption{An illustration of the equipments for data collection. Each participant wears a CGM device to record blood glucose concentration and uses a smartphone to collect external factors. }
  \label{fig:experiment_case}
\end{figure}

\textbf{Datasets.}
We validate \sysname on a dataset of $112$ participants ($35$ non-diabetes, $38$ type I diabetic patients and $39$ type II diabetic patients) collected during July 2016 to January 2017.
Each participant is equipped with (1) a WAVEGUIDER \emph{U-Tang} CGM device~\cite{bib:CGM_wave} to record blood glucose concentration every $3$ minutes and (2) a smartphone with \sysname installed to collect external factors either automatically (activities and sleep quality) or manually (food, drug, and insulin intake).
All participants agree to take measurements (\ie wear the CGM device and use \sysname to record external factors) for at least $6$ days, which is a disposable usage duration of enzyme of the CGM.
\figref{fig:experiment_case} illustrates an example of data collection from a user.
In total we obtain 762639 samples of blood glucose concentration and the corresponding external factors covering around 38132 hours.
In brief, we collect the following categories of data:
\begin{itemize}
  \item
  \textbf{Meta information.}
  We record basic personal data including gender, age, weight and health status to cover a wide range of users.
  \tabref{tab:parcitipant} summarizes the basic information of the participants.
  \item
  \textbf{Blood glucose measurements.}
  We collect blood glucose measurements using commercial CGM devices for 6 to 30 days as labeled data.
  \tabref{tab:bgdata} summarizes the blood glucose measurements in our evaluation.
  \item
  \textbf{External factor measurements.}
  During measurements of blood glucose concentration, each participant manually inputs the times of their daily meal, drug and insulin intake.
  \sysname automatically records activity levels and sleep quality as in \secref{subsec:external}.
  \figref{fig:experiment_case} shows the user interfaces to record external factors.
\end{itemize}

\begin{table}
  \centering
  \caption{Summary of participant information.}
  \label{tab:parcitipant}
  \subfloat[]{%
  \begin{tabular}{cc}
  \toprule
  \textbf{Age (year)} & \textbf{\# User} \\
  \midrule
  15-24 & 8 \\
  25-34 & 17 \\
  35-44 & 24 \\
  45-54 & 29 \\
  55-64 & 34 \\
  \bottomrule
  \end{tabular}}%
  \quad% --- set horizontal distance between tables here
  \subfloat[]{%
  \begin{tabular}{ccc}
  \toprule
  \textbf{Weight (kg)} &\textbf{BMI ($kg/m^2$)}\cite{bib:world2013bmi} &\textbf{\# User} \\
  \midrule
  Underweight & (0,  18.5) & 18 \\
  Normal weight & [18.5,  25) & 31 \\
  Overweight &  [25,  30) &41 \\
  Obese &  [30, +$\infty$) & 22 \\
  \bottomrule
  \end{tabular}}%
  \quad%
%  \%subfloat[]{%
%  \begin{tabular}{cc}
%  \toprule
%  \textbf{Status} & \textbf{\# User} \\
%  \midrule
%  Non-diabetes & 35 \\
%  Type I & 38 \\
%  Type II & 39 \\
%  \bottomrule
%  \end{tabular}}
%  \quad%
  \subfloat[]{%
  \begin{tabular}{cc}
  \toprule
  \textbf{Gender} & \textbf{\# User} \\
  \midrule
  Male & 57 \\
  Female & 55\\
  \bottomrule
  \end{tabular}}
\end{table}

\begin{table}
  \centering
  \caption{Summary of blood glucose measurements.}
  \label{tab:bgdata}
  \subfloat[]{%
  \begin{tabular}{cc}
  \toprule
  \textbf{Duration (days)} & \textbf{\# User} \\
  \midrule
  6-10 & 48 \\
  11-15 & 24 \\
  16-20 & 20 \\
  21-25 & 13 \\
  26-30 & 7 \\
  \bottomrule
  \end{tabular}}%
  \qquad% --- set horizontal distance between tables here
  \subfloat[]{%
  \begin{tabular}{cc}
  \toprule
  \textbf{Blood Glucose} & \textbf{\# Sample} \\
  \midrule
  Level 1 & 75369 \\
  Level 2 & 293530 \\
  Level 3 & 235686 \\
  Level 4 & 158054 \\
  Total & 762639 \\
  \bottomrule
  \end{tabular}}%
\end{table}

%\begin{table}[]
%  \centering
%  \caption{Summary of experimental settings.}
%  \label{tab:dataset}
%  \begin{tabular}{clccclcccccl}
%  \hline\hline
%  \multicolumn{12}{c}{\textbf{Blood Glucose}}                                                                                                                                                                                               \\ \hline
%  \multicolumn{2}{l}{\textbf{\cellcolor[gray]{0.8}Blood Level}} & \multicolumn{2}{r}{\textbf{\cellcolor[gray]{0.8}Level 1}} & \multicolumn{2}{c}{\textbf{\cellcolor[gray]{0.8}Level 2}} & \multicolumn{2}{c}{\textbf{\cellcolor[gray]{0.8}Level 3}} & \multicolumn{2}{c}{\textbf{\cellcolor[gray]{0.8}Level 4}} & \multicolumn{2}{c}{\textbf{\cellcolor[gray]{0.8}Total}} \\
%  \multicolumn{2}{l}{Number of Sample}     & \multicolumn{2}{c}{75369}            & \multicolumn{2}{c}{293530}           & \multicolumn{2}{c}{235686}           & \multicolumn{2}{c}{158054}           & \multicolumn{2}{c}{762639}         \\ \hline\hline
%  \multicolumn{12}{c}{\textbf{Experimental Duration}}                                                                                                                                                                                          \\ \hline
%  \multicolumn{2}{l}{\textbf{\cellcolor[gray]{0.8}Days}}        & \multicolumn{2}{c}{\textbf{\cellcolor[gray]{0.8}6-10}}    & \multicolumn{2}{c}{\textbf{\cellcolor[gray]{0.8}11-15}}   & \multicolumn{2}{c}{\textbf{\cellcolor[gray]{0.8}16-20}}   & \multicolumn{2}{c}{\textbf{\cellcolor[gray]{0.8}21-25}}   & \multicolumn{2}{c}{\textbf{\cellcolor[gray]{0.8}26-30}} \\
%  \multicolumn{2}{l}{Number of Users}      & \multicolumn{2}{c}{48}               & \multicolumn{2}{c}{24}               & \multicolumn{2}{c}{20}               & \multicolumn{2}{c}{13}               & \multicolumn{2}{c}{7}              \\
%  \hline\hline
%  \multicolumn{12}{c}{\textbf{Summary of participant data}}                                                                                                                                                                                      \\ \hline
%  \multicolumn{2}{l}{\textbf{\cellcolor[gray]{0.8}Age}}         & \multicolumn{2}{c}{\textbf{\cellcolor[gray]{0.8}15-24}}   & \multicolumn{2}{c}{\textbf{\cellcolor[gray]{0.8}25-34}}   & \multicolumn{2}{c}{\textbf{\cellcolor[gray]{0.8}35-44}}   & \multicolumn{2}{c}{\textbf{\cellcolor[gray]{0.8}45-54}}   & \multicolumn{2}{c}{\textbf{\cellcolor[gray]{0.8}55-70}} \\
%  \multicolumn{2}{l}{Number of Users}      & \multicolumn{2}{c}{8}                & \multicolumn{2}{c}{17}               & \multicolumn{2}{c}{24}               & \multicolumn{2}{c}{29}               & \multicolumn{2}{c}{34}             \\
%  \multicolumn{2}{l}{\textbf{\cellcolor[gray]{0.8}Weight}}      & \multicolumn{2}{c}{\textbf{\cellcolor[gray]{0.8}30-44}}   & \multicolumn{2}{c}{\textbf{\cellcolor[gray]{0.8}45-54}}   & \multicolumn{2}{c}{\textbf{\cellcolor[gray]{0.8}55-64}}   & \multicolumn{2}{c}{\textbf{\cellcolor[gray]{0.8}65-74}}   & \multicolumn{2}{c}{\textbf{\cellcolor[gray]{0.8}75-90}} \\
%  \multicolumn{2}{l}{Number of Users}      & \multicolumn{2}{c}{18}               & \multicolumn{2}{c}{21}               & \multicolumn{2}{c}{32}               & \multicolumn{2}{c}{22}               & \multicolumn{2}{c}{19}             \\
%  \multicolumn{4}{l}{\textbf{\cellcolor[gray]{0.8}Gender}}      & \multicolumn{3}{c}{\textbf{\cellcolor[gray]{0.8}Male}}                                                               & \multicolumn{5}{c}{\textbf{\cellcolor[gray]{0.8}Female}}                                                          \\
%  \multicolumn{4}{l}{Number of Users}      & \multicolumn{3}{c}{57}                                                                          & \multicolumn{5}{c}{55}                                                                       \\ \hline\hline
%  \multicolumn{12}{c}{\textbf{User Health Status}}                                                                                                                                                                                          \\ \hline
%  \multicolumn{3}{l}{\textbf{\cellcolor[gray]{0.8}Health Status}}                   & \multicolumn{3}{c}{\textbf{\cellcolor[gray]{0.8}Health}}                     & \multicolumn{3}{c}{\textbf{\cellcolor[gray]{0.8}Type I}}                      & \multicolumn{3}{c}{\textbf{\cellcolor[gray]{0.8}Type II}}                  \\
%  \multicolumn{3}{l}{Number of Users}                          & \multicolumn{3}{c}{35}                                  & \multicolumn{3}{c}{38}                                   & \multicolumn{3}{c}{39}                                \\ \hline
%  \end{tabular}
%\end{table}

\textbf{Ground Truth.}
We use the blood glucose concentrations collected by the CGM device as ground truth \footnote{While clinical studies report that the precision and accuracy of commercial CGM devices still need improving~\cite{bib:MEP08:Do, bib:JDST10:Vaddiraju}, they are sufficient as ground truth for the four normal and abnormal blood glucose levels.}.

\textbf{Metrics.}
We mainly adopt precision, recall and accuracy~\cite{prf1} to quantify the performance of \sysname.
%
%
%\begin{table}[]
%\centering
%\caption{The details of dataset}
%\label{The details of dataset}
%\begin{tabular}{|l|c|c|c|c|l|}
%\hline
%\textbf{Blood Level}                  & \textbf{Level 1} & \textbf{Level 2} & \textbf{level 3} & \textbf{Level 4} & \textbf{Total}         \\ \hline
%\multicolumn{1}{|c|}{\textbf{Number}} & 75369            & 293530           & 235686           & 158054           & \multicolumn{1}{c|}{762639} \\ \hline
%\end{tabular}
%\end{table}


\subsection{Inference Accuracy}
\subsubsection{Overall Inference Accuracy}
Since all participants collected both measurements of CGM and external factors for at least 6 days, \ie a normal usage duration of enzyme in the CGM, we use measurements during the former 5 days for training and the rest for testing.
\tabref{tab:confusion_matrix} shows the overall performance of \sysname.
All results are averaged over the testing data.
As shown, the recalls and the precisions for all the 4 blood glucose levels are above 79\% and 73\%, respectively.
In particular, the recalls for Level 1 (low blood glucose) and Level 4 (high blood glucose) are 83.13\% and 85.23\%, even though the training data for Level 1 and Level 4 only take up 9.88\% and 20.72\% of the entire training set.
This result shows that \sysname can accurately infer low/high blood levels even with an imbalanced training dataset.
Overall, \sysname yields an accuracy of 82.14\%, showing a promising performance to track blood glucose levels.

\begin{table}[h]
  \centering
  \caption{Confusion matrix of \sysname.}
  \label{tab:confusion_matrix}
  \begin{tabular}{|c|c|c|c|c|l|l|}
  \hline
  \multirow{2}{*}{\textbf{\begin{tabular}[c]{@{}c@{}}Ground\\ Truth\end{tabular}}} & \multicolumn{4}{c|}{\textbf{Inference}}                                                                                 & \multicolumn{2}{l|}{\multirow{2}{*}{}}                                                            \\ \cline{2-5}
                                                                                 & Level 1                      & Level 2                      & Level 3                      & Level 4                      & \multicolumn{2}{l|}{}                                                                             \\ \hline
Level 1                                                                          & \cellcolor[gray]{0.8}62657                        & 5521                         & 3672                         & 3519                         & 83.13\%                             & \multirow{4}{*}{\rotatebox{90}{\textbf{Recall}} }                           \\ \cline{1-6}
Level 2                                                                          & 16346                        &  \cellcolor[gray]{0.8}240584                       & 27563                        & 9037                         & 81.96\%                             &                                                             \\ \cline{1-6}
Level 3                                                                          & 2660                         & 30905                        & \cellcolor[gray]{0.8}188472                       & 13649                        & 79.97\%                             &                                                             \\ \cline{1-6}
Level 4                                                                          & 3443                         & 5620                         & 14278                        & \cellcolor[gray]{0.8}134713                       & 85.23\%                             &                                                             \\ \hline
\multicolumn{1}{|l|}{\multirow{2}{*}{}}                                          & \multicolumn{1}{l|}{73.62\%} & \multicolumn{1}{l|}{85.12\%} & \multicolumn{1}{l|}{80.55\%} & \multicolumn{1}{l|}{83.72\%} & \multicolumn{2}{l|}{\multirow{2}{*}{\begin{tabular}[c]{@{}l@{}}Accuracy:\\ 82.14\%\end{tabular}}} \\ \cline{2-5}
\multicolumn{1}{|l|}{}                                                           & \multicolumn{4}{c|}{\textbf{Precision}}                                                                                 & \multicolumn{2}{l|}{}                                                                             \\ \hline
\end{tabular}
\end{table}


\subsubsection{Inference Result Analysis}
\label{subsec:predict_result_analysis}
To understand the inference accuracy and the risks of different types of errors in the context of blood glucose management, we classify the inference results based on the Clarke Error Grid Analysis (CEGA) \cite{bib:DTT05:Clarke}.
The analysis classifies the inference results into correct event (Type A) and different types of errors (Type B to Type E) with increasing levels of severity.
For instance, Type B errors are those that will not lead to inappropriate treatments, while Type E errors can lead to wrong treatment.
\tabref{predict_results} summarizes the percentages of each type of results.
As shown, \sysname will not cause inappropriate treatment (Type A and B) in almost 90\% of the cases.
It may lead to unnecessary worries or treatment (Type C) in 5.47\% of the cases.
In fewer than 5\% of the cases, \sysname will miss an abnormal blood glucose event (Type D) or confuse treatment (Type E).
Therefore, \sysname is suitable as an temporal alternative for CGM devices.
However, we do not recommend \sysname for extended duration of usage for patients serious diabetics, who need regular blood glucose management.

\small
\begin{table}[h]
\centering
\caption{Inference Result Analysis}
\label{predict_results}
\begin{tabular}{|c|l|c|}
\hline
\textbf{Type of Result} & \multicolumn{1}{c|}{\textbf{Explanation of Result}}                                                                                                                                                                                                & \textbf{Percentage} \\ \hline
Type A                  & \begin{tabular}[c]{@{}l@{}}The inference value is consistent with the true value.\\ (\ie the inference blood glucose level is correct.)\end{tabular}                                                                                             & 82.14\%             \\ \hline
Type B                  & \begin{tabular}[c]{@{}l@{}}The inference result would not lead to inappropriate treatment. \\ (\ie Level 2 is predicted as Level 3, or vice-versa.)\end{tabular}                                                                                 & 7.67\%              \\ \hline
Type C                  & \begin{tabular}[c]{@{}l@{}}The inference result will lead to unnecessary treatment. \\ (\emph{i.e.}, Level 2 is predicted as Level 1/4,or Level 3 is predicted as Level 1/4.)\end{tabular}                                                                       & 5.47\%              \\ \hline
Type D                  & \begin{tabular}[c]{@{}l@{}}Fail to detect hypoglycemia or hyperglycemia.\\ (\ie Level 1/4 are predicted as Level 2/3.)\end{tabular} & 3.81\%              \\ \hline
Type E                  & \begin{tabular}[c]{@{}l@{}}The predicted results that would confuse treatment by mistaking hypoglycemia\\ for hyperglycemia or vice-versa.\\ (\ie Level 1 is predicted as Level 4, and vice-versa.)\end{tabular}                                           & 0.91\%              \\ \hline
\end{tabular}
\end{table}


\subsubsection{Temporal View of Inference Results}
\figref{fig:pre_gt} plots the example inference results of \sysname of three participants (one non-diabetic, one Type I diabetic patient, and one Type II diabetic patient) throughout a day.
The errors are depicted at the bottom of each figure.
As shown, the true blood glucose levels vary during the day after important daily activities such as food intake (5:00, 11:30 and 19:00 for the non-diabetic user; 6:00 and 16:50 for the type I diabetic user; 6:00, 12:50 and 17:45 for the type II user), insulin injection (7:30 for the type I diabetic user), drug intake (15:10 for the type II user) and exercises (15:30 for the type II user), indicating the importance of external factors.
The blood glucose levels inferred by \sysname also match the true blood glucose levels most of the time, which validates the effectiveness of \sysname during various daily activities.

Most errors mistake adjacent blood glucose levels, and usually occur during the transition of two blood glucose levels (\eg from 6:00 to 6:30 for the type II user), or in case of sudden blood glucose concentration change (\eg at 2:30 for the non-diabetic user and at 0:30 for the type I user).
Errors during blood glucose transitions are mainly caused by the delays to measure the external factors.
Errors in case of sudden blood glucose changes occur because the sudden changes in blood glucose concentration may not immediately result in sudden changes in the external factors.
Nevertheless, both errors tend to occur for a short duration of time, which will not lead to risky emergencies.


%For the former one, it is mainly due to the external factors tracked by \modelname have just changed for a short time, resulting in a little temporal delay to follow the true values. This type of error, however, can be revised quickly.
%For the latter one, the period of sudden blood glucose fluctuation lasts too short for \sysname to detect the dynamics of external factors. Nevertheless, the duration of this kind of error is too short, which hardly harm the users.

%\textcolor[rgb]{1.00,0.00,0.00}{Moreover, we can also learn most of errors belongs to the Type B/C  rather than Type D/E (serious error) in the \tabref{predict_results}, which are also accordant with the conclusion in \secref{subsec:predict_result_analysis}.}

\begin{figure}[h]
  \centering
  \includegraphics[width=0.8\columnwidth]{./img/pred_vs_gt2.pdf}
  \caption{Traces of blood glucose level inference results throughout a day.}
  \label{fig:pre_gt}
\end{figure}

\subsection{Model Comparison}

\subsubsection{Effectiveness of Multi-division Framework}
To demonstrate the effectiveness of the multi-division framework in making full use of the training dataset, we evaluation \modelname from two perspectives.

\fakeparagraph{Layer contribution analysis}
To evaluate the effect of different layers, we conduct blood glucose level inference with three combinations of layers.
\begin{itemize}
  \item
  \emph{Deep dynamic layer.}
  Training without considering differences in groups and only output a general model.
  \item
  \emph{Grouped input layer + deep dynamic layer.}
  Learn group-specific feature representations but ignore per-person characteristics in the output.
  \item
  \emph{Grouped input layer + deep dynamic layer + personalized output layer (\modelname).}
  Efficiently learn features from different groups and output personalized inference results.
\end{itemize}
\figref{fig:cmp_model} plots the comparison results of the three combinations.
As shown, both the precisions and recalls increase with more layers, with an improvement of 21.13\% in average precision and 18.57\% in average recall, respectively.
Moreover, the standard deviations drop remarkably from 17.25\% to 10.25\%  of average precision, and from 20.75\% to 10.75\% of average recall.
The results demonstrate the effectiveness of \modelname, which learns representative features from the same groups and considers individual differences in blood glucose level inference.

\begin{figure}[h]
  \centering
  \includegraphics[width=0.9\columnwidth]{./img/CMP_Models1.pdf}
  \caption{Performance of layer combinations.}
  \label{fig:cmp_model}
\end{figure}

\fakeparagraph{Comparison of data sharing schemes}
To demonstrate the benefits of sharing data and knowledge among groups and users, we compare \modelname with other learning frameworks with different data sharing schemes.
\begin{itemize}
  \item \emph{General Learning.}
  All the training data are directly fed into the model (\ie deep RNN) for training indifferently.
  General learning results in a \emph{generic} model that assumes universal correlations between all inputs and the blood glucose levels.
  \item \emph{Group Learning.}
  The data of users belonging to a same group are fed into a model (\ie deep RNN) for training.
  Three separate models are obtained for three groups (\ie non-diabetic, type I and type II diabetic).
  The group learning results in a \emph{group} model that shares the general characteristics of users within the same group but without data sharing among users in different groups.
  \item \emph{Single Learning.}
  We train a different model (\ie deep RNN) for each individual participant by feeding his/her own measurements into the model.
  Single learning results in a \emph{personalized} model without sharing data and learning knowledge from measurements of other participants.
\end{itemize}

\figref{fig:cmp_multi_division} shows the overall precisions and recalls of our \modelname as well as \emph{General learning}, \emph{Group learning} and \emph{Single learning}.
As shown, our multi-divisional learning framework (\modelname) performs best among the four learning approaches with an average precision of 80.75\% and an average recall of 82.57\%.
It also yields the lowest standard deviations (17.18\% of average precision and 17\% of average recall).
The results show that \modelname is both effective and stable in blood glucose level inference.

General learning treats each sample of training data equally, and ignores the individual differences, so it performs poorly in most cases.
Conversely, single learning approach encodes the individual characteristics but suffers from lacking of user-specific training dataset. Even though group learning learns the similarities of users within the same group, it ignores inter-person physiological differences.
\modelname combines the advantages of these three learning approaches, which makes better use of the limited training data by sharing measurements among users and preserves user-specific characteristics via the personal learning layer.

%General learning performs slightly better than single learning for Level 2 and 3 (normal blood glucose levels), partly because the correlations between the inputs and normal blood glucose levels are relatively consistent for most people, while single learning suffers from lack of training data as it only uses user-specific data.
%Conversely, single learning achieves higher precision and recall than general learning for Level 1 and 4 (abnormal blood glucose levels), partly because there are notable inter-person differences in the correlations between the inputs and abnormal blood glucose levels.
%That is, the reasons for abnormal blood glucose levels can vary from person to person.


\begin{figure}[h]
  \centering
  \includegraphics[width=0.9\columnwidth]{./img/performance_of_multi_division.pdf}
  \caption{Performance comparison of different data sharing schemes.}
  \label{fig:cmp_multi_division}
\end{figure}


\subsubsection{Effectiveness of Deep, Multi-division Learning}
To demonstrate the effectiveness of adopting deep learning algorithms over conventional shallow learning algorithms, we compare our \modelname with the following algorithms.
\begin{itemize}
  \item
  \textbf{Gradient Boosting (GB).}
  GB \cite{bib:friedman2002stochastic} generates a prediction model by combining many weak classifiers into a stronger classification committee.
  We use AdaBoost procedure implemented in the fastAdaboost package to combine basic tree classifiers for ensemble learning.
  %We vary the maximum tree depth from 10 to 50 by factors of ten.
  %The number of boosting iterations is varied from 100 to 500 by a step size of 50.
  \item
  \textbf{Support Vector Machine (SVM).}
  SVM \cite{bib:wang2005support} bases on the idea of ��optimal separating hyperplane�� that maximizes the separation margin of two data groups (classes).
  Due to this construction, it usually generalizes well, and its dual form is a quadratic programing that can be easily incorporated with kernels.
  We train the Gaussian kernel SVM classifier with the kernlab package, which implements the sequential minimal optimization algorithm.
  %We vary the kernel width from $2^{-5}$ to $2^4$ with a factor of 2.
  %We pick the penalty parameter from the set $\{10^i | i = -3, 0.5, 2\}$.
  To eliminate scale/location discrepancies among input variables, all features are normalized before being used in the training phase.
  \item
  \textbf{Hidden Markov model (HMM).} \cite{bib:rabiner1986introduction}
  Being a classical example of dynamic Bayesian networks, HMM assumes that the observed process is driven by a hidden (unobserved) Markovian process. Simple as it is, HMM is widely used in signal processing and time series analysis due to its flexibility and tractability. In addition, it has close tights with optimal filtering and state estimation. In our implementation, HMM is learned with the classic Viterbi algorithm. 
  \item
  \textbf{Artificial neural network (ANN).}\cite{bib:wang2003artificial}
  We also included the classical ANN as a baseline, simply to justify the benefit of ``structure engineering'' from \modelname. The ANN under comparison contains a single input layer, three hidden layers, and an output layer. The training of ANN is done by using the stochastic gradient descend algorithm implemented in Tensorflow.  
  \item
  \textbf{Random Forest (RF).}
  As another ensemble method, RF~\cite{bib:liaw2002classification} combines many simple decision trees together and output the mode of classes for prediction.
  To avoid correlation among base trees, random set of features are selected in the splitting process when constructing each decision tree.
  For implementation, we adopt the conditional inference tree algorithm in the Party package.
  %The total number of trees is tuned from 100 to 1000, and the maximum tree depth from 10 to 50.
  %The splitting threshold is also varied from 0.1 to 0.9 with 0.1 intervals for cross validation.
  \item
  \textbf{Gaussian Processes (GP).}
  Instead of directly parameterizing a latent function for classification, GP~\cite{bib:rasmussen2006gaussian} models it with a generic Gaussian process.
  The posterior of the process is updated with training data set, and is ``squashed'' through a logistic function for classification.
  We implement GP with the kernlab package, which includes several approximation algorithms for acceleration.
  %We use the radial basis kernel and vary the kernel width from 2$^{-5}$ to 2$^4$ with an incremental factor of 2$^{0.5}$.
\end{itemize}

\figref{fig:cmp_models} illustrates the results.
Apparently, \modelname achieves best performance on both precisions and recalls.
More specifically, it outperforms the runner-up by at least 20\% in terms of precision, and yields much better recalls for the categories of interest, i.e., level 1 and level 4. Among those baselines, it appears that no method could dominate the others, except that HMM performs slightly better in terms of recall score. This is mainly because HMM is the only method among baselines that incorporates temporal dependence. However compared to \modelname which is able to describe multi-scale dynamics, overall HMM is still much worse.  The overwhelming performance of \modelname is somewhat expected, as those classical baselines either ignore the multi-scale dynamics of the observed data, or does not allow information sharing among available data from grouped users. The above observation further justifies the effort of adopting valuable knowledge about the application, for the design of new machine learning paradigms. 
\begin{figure}[h]
  \centering
  \includegraphics[width=0.9\columnwidth]{./img/Model_CMP.pdf}
  \caption{Performance comparison with shallow learning algorithms.}
  \label{fig:cmp_models}
\end{figure}




%It was then used to quantify the clinical accuracy of blood glucose estimates generated by meters as compared to a reference value.

\subsection{Micro-benchmarks}
\subsubsection{Effectiveness of Features}
\tabref{tab:features} shows the average precisions and recalls for all the 4 blood glucose levels with different combinations of features.
By combining physiological features ($X_{P}$) with temporal features ($X_{T}$), the average precision and recall of the 4 blood glucose levels improve by 31.38\% and 41.48\%, respectively.
Specifically, the physiological features ($X_{P}$) and the historical blood glucose trend ($X_{T_1}$) bring in the most notable improvement in detecting abnormal blood glucose events (Level 1 and Level 4).
This is because the physiological model can well describe the impact of external factors (\eg food and exercise), and the historical trend enables to track the individual temporal dynamics of average blood glucose concentrations in recent time. 

\begin{table}[h]
  \small
  \centering
  \caption{Effectiveness of features.\TODO{no $X_{T_2}$ now.}}
  \label{tab:features}
  \begin{tabular}{|c|c|c|c|c|c|c|c|c|}
  \hline
                                   & \multicolumn{2}{c|}{\textbf{Level 1}}                     & \multicolumn{2}{c|}{\textbf{Level 2}} & \multicolumn{2}{c|}{\textbf{Level 3}}                     & \multicolumn{2}{c|}{\textbf{Level 4}}                     \\ \hline
  \textbf{Features}                  & \textbf{Precision} & \multicolumn{1}{l|}{\textbf{Recall}} & \textbf{Precision}  & \textbf{Recall} & \textbf{Precision} & \multicolumn{1}{l|}{\textbf{Recall}} & \textbf{Precision} & \multicolumn{1}{l|}{\textbf{Recall}} \\ \hline
  $X_{P}$                            & 43.37$\%$               & 32.82$\%$                                 & 46.03$\%$                & 39.10$\%$            & 51.79$\%$               & 48.95$\%$                                 & 56.30$\%$               & 43.49$\%$                                 \\ \hline
  $X_{P}$+$X_{T_1}$                   & 51.97$\%$               & 58.11$\%$                                 & 60.42$\%$                & 58.90$\%$            & 63.35$\%$               & 53.59$\%$                                 & 69.82$\%$               & 65.16$\%$                                 \\ \hline
  $X_{P}$+$X_{T_1}$+$X_{T_2}$          & 64.60$\%$               & 73.08$\%$                                 & 69.87$\%$                & 61.23$\%$            & 74.33$\%$               & 67.81$\%$                                 & 76.64$\%$               & 72.32$\%$                                 \\ \hline
  $X_{P}$+$X_{T_1}$+$X_{T_2}$+$X_{T_3}$ & 73.62$\%$   & 83.13$\%$                                 & 85.12$\%$               & 81.96$\%$            & 80.55$\%$   & 79.97$\%$
  & 83.72$\%$               & 85.23$\%$                                  \\ \hline
  \end{tabular}
\end{table}


\subsubsection{Impact of amount of training samples}
In this experiment, we evaluate the performance of \sysname with increasing numbers of training samples.
Since the duration of measurements for each participant varies from 6 to 30 days, we use measurements of 5 to 25 days for training, and the rest for testing.
We keep the measurements for training but exclude them for testing if the duration of the measurements is short.
For example, if the user's measurements last for 7 days, we use his measurement to evaluate the performance of using 5 days of training data, and test on the measurements of the remaining 2 days.
However, when evaluating the performance with 10 days of training data, we only use the 7 days of measurements for training, but not for testing.

\begin{figure}[h]
  \centering
  \includegraphics[width=0.9\columnwidth]{./img/performance_under_days1.pdf}
  \caption{Impact of increasing amount of training samples.}
  \label{fig:per_under_train_days}
\end{figure}

\figref{fig:per_under_train_days} illustrates the results for all the 4 blood glucose levels.
The results are averaged over all testing samples as in previous evaluations.
As expected, the precisions and recalls for all the 4 blood glucose levels improve smoothly with the increase of training samples.
The results verify that the challenge (and our motivation to adopt a multi-task learning framework) is the lack of training data.
Note that \sysname is not a replacement of the current CGM devices, but rather, a complement when CGM devices are uncomfortable or inconvenient to wear.
Therefore we envision the training dataset will grow gradually after wearing the CGM device multiple times (at least for diabetes patients), and the overall accuracy will also improve over time as a result.



\subsubsection{Impact of Temporal Gaps}
The blood glucose concentration is correlated with the previous blood glucose levels because of the control loop of the glucose metabolism~\cite{bib:TBE07:Dalla, bib:PE04:Hovorka, bib:IJNMBE16:Oviedo}.
Since \sysname does not rely on the previous blood glucose level as an input, it is natural that the accuracy of \sysname will degrade if there is a long gap between the training and the testing datasets (\ie the training dataset can be outdated).

\begin{figure}[h]
  \centering
  \includegraphics[width=0.9\columnwidth]{./img/Performance_gap1.pdf}
  \caption{Impact of temporal gaps between the training and testing datasets.}
  \label{fig:per_under_various_pred_days}
\end{figure}

\figref{fig:per_under_various_pred_days} plots the overall performance by training using the same 5 days of measurements, and testing on measurements collected on the 6-10th, 11-15th, 16-20th, 21-25th, and 26-30th days, respectively.
As expected, both the precisions and recalls drop moderately with the increase of temporal gaps between the training and the testing datasets, with a maximum decrease of 6.73\% and 7.02\% in average precision and recall after 21-25 days.
Note that \sysname is not designed as a replacement of the commercial CGM devices, but rather a ubiquitous temporary alternative when CGM devices are uncomfortable or inconvenient to wear.
From the results, we recommend \sysname users to put on the CGM device to monitor the blood glucose at least every three weeks.
The data sampled by the CGM will automatically feed into \sysname for a model retraining.




%\subsubsection{Detection Accuracy for Different Groups}
%Fig.~\ref{fig:cmp_groups} illustrates the results of each groups.
%
%\TODO{redraw the figures, no need to include general and personal learning here. show the performance for different health status, gender, age groups, weight groups, \etc, explain why \sysname works better for certain groups, or works well for all groups}.
%
%We also apply general learning approach on the users in same group, and compare the prediction performance of \modelname. Fig.~\ref{fig:cmp_groups}
%shows the results. As is shown, \modelname outperforms the general learning methods in each group, especially the performance of level 1 and level 4. It mainly results by two reasons. On the one hand, the limitation of blood glucose data in each group weakens the capability of temporal dynamic characteristics. On the other hand, the imbalanced distribution of blood glucose data of one group also low the performance down. For example, much more data of level 4 and much less data of level 1 in group 3 (type II diabetes) low down the recall of level 1 and the precision of level 4. It is even hard to be solved by the cost sensitive approach.


%
\section{Discussion}
\label{sec:conclusion}
In our work, we use the continuous glucose monitor (CGM) to collect the groudtruth. Its measurement has bias and delay. 

\section{Conclusion}
\label{sec:conclusion}

Designing a pervasive and real-time blood glucose tracking system is crucial yet challenging. Existing devices either limited by the discontinuous usage (\eg finger pricking) or life interference (\ie continuous blood glucose monitor), preventing users measuring their blood glucose value regularly.
The traditional inference models, however, suffer from the sparse data, resulting in the issue of low accuracy.
In this paper, we implemented \sysname, a mobile framework tracking personal blood glucose level in real time.
By adopting a novel statistical model \modelname, \sysname gets rid
of the sparsity of individual blood glucose data, and then encodes the relationships between external factors and blood glucose levels comprehensively. The inference results could benefit users in learning their
blood glucose alternation without interference, and further guiding their healthy behaviors (\eg the amount of food, drug and insulin intake, the sleep duration). We implemented \sysname on iOS platform and examined its performance on 112 participants over 7 months. The results show that \modelname outperforms the traditional statistical models, and achieves desirable monitor accuracy, making \sysname satisfy in daily use.


\bibliographystyle{ACM-Reference-Format}
\bibliography{sample-ubicomp}
\end{document}
%%% Local Variables:
%%% mode: latex
%%% TeX-master: t
%%% End:


